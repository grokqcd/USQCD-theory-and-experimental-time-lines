\documentclass[prd,showpacs,showkeys,preprintnumbers,floatfix,
nofootinbib%,superscriptaddress
]{revtex4-2}
\usepackage[utf8]{inputenc}
\usepackage{graphicx}
\usepackage{amsmath}
\usepackage{amsfonts} 
\usepackage{hyperref}
% \usepackage[ruled,vlined]{algorithm2e}
\hypersetup{
    colorlinks   = true,
    citecolor    = blue
}


\begin{document}

%\documentclass[12pt]{article}
%\usepackage{graphicx} % Required for inserting images
%\usepackage{hyperref}
%\setlength{\textwidth}{480pt}
%\setlength{\textheight}{650pt}
%\setlength{\oddsidemargin}{0pt}
%\setlength{\topmargin}{-60pt}

\title{USQCD theory and experimental time lines}
\author{USQCD Collaboration}
\date{April 2023}


\maketitle
\tableofcontents
\section{Intensity Frontier}\label{sec:intensity}

% Tom Blum
\subsection{Muon $g-2$}
\begin{itemize}
    \item[Motivation.] The muon anomalous magnetic moment, or $g-2$, will be measured by the E989 experiment at Fermilab to the one-per=mille precision level and thus offers an exceptional opportunity to test the standard model if theory can match this precision.
    \item[Long term goal.] The Muon $g-2$ Theory Initiative aims to provide a consensus theory value at the same level of precision as the E989 experiment. Only QCD, or hadronic, contributions need improvement to reach this goal.
\item[Method.] Lattice  calculations of correlation functions in QCD for the hadronic vacuum polarization (HVP) (two-point) and hadronic light-by-light scattering (HLbL) (four-point) are combined with perturbative QED at $O(\alpha^2)$ and $O(\alpha^3)$, respectively, to obtain the leading hadronic contributions.
\item[Timeline:]
%\begin{itemize}
    \item[2021] Muon $g-2$ Theory Initiative whitepaper~\cite{Aoyama:2020ynm} released with consensus Standard Model Value based on data-driven HVP and data-driven+lattice HLbL contributions.
    \item[2021] E989 announces first results with 0.48 ppm precision. Combined with BNL 821 (0.54 ppm), yields 4.2 standard deviation discrepancy with theory.
    \item[2021] BMW collaboration computes HVP contribution to 0.75\%, consistent with experiment. 
    \item[2022] USQCD groups Fermilab Lattice/HPQCD/MILC and RBC/UKQCD, $\chi$QCD, Aubin {\it et al.}, and several groups from Europe obtain agreement on the HVP intermediate window which is four standard deviations discrepant with the data-driven value.
    \item[2023] Several groups to update total HVP contribution, including isospin corrections, with sub-percent errors.
    \item[2023] RBC updated HLbL contribution, reaching 12\% error.
    \item[2023] E989 to announce run 2 and 3 combined result, improving precision by a factor of two.
    \item[2024] RBC to reduce error on HLbL contribution by a factor of two.
    \item[2024-25] RBC/UKQCD and FHM reach 1-2 per-mille goal on HVP contribution. 
    \item[2025-26] E989 releases final result with expected precision of $\sim 1.4$ ppm
%\end{itemize}
\end{itemize}

% Carleton Detar
\subsection{$B$-meson anomalies and CKM unitarity tests}
\begin{itemize}
    \item[Motivation.] $B$-flavor physics offers excellent opportunities for stringent tests of the Standard Model, tests that may lead to the discovery of new physics. In conjunction with refined experimental results, precision lattice-QCD calculations yield precise values of the CKM matrix elements, allowing checks of CKM unitarity and tests of lepton universality. Measurements of rare semileptonic $B$-meson decays (such as $B\to\pi\ell^+\ell^-$, $B\to K\ell^+\ell^-$) are promising channels for new-physics searches because their rates are suppressed in the Standard Model.
    \item[Long term goal.] Lattice-QCD calculations go hand-in-hand with experiment.  Thus the long-term goal is to keep pace with the improving experimental precision of each quantity under study.
    \item[Method.] Most calculations work with a combination of hadronic two- and three-point functions, measured at several lattice spacings and quark masses, and with a variety of current operators. 
\item[Timeline:]
%\begin{itemize}
    \item[2014] Calculation of leptonic $B$, $B_s$, $D$, and $D_s$ decay constants to sub-percent accuracy.
    \item[2015] First measurement by CMS/LHCb of the decay $B_s \to \mu^+\mu^-$.
    \item[2021] Calculation of $R(D*)$ to 4\%.
    \item[2022] Precise calculation of $|V_{cs}|$ with a QCD error comparable to the current experimental precision and of $|V_{cd}|$ with a QCD error twice that of experiment.
    \item[2024] Calculation of $|V_{cb}|$ to 1\% via $B\to D\ell\nu$ and $|V_{ub}$ to 2\% via $B\to \pi\ell\nu$.
    \item[2024] Calculation of $B\to\pi \ell^+\ell^-$ and $B\to K \ell^+\ell^-$ to 2\%.
    \item[2027] Calculation of $|V_{cb}|$ to the sub-percent level via $B\to D^*\ell\nu$.
    \item[2027] Calculation of $R(D*)$ to 1\%.
    \item[??] Belle II
    \item[??] LHCb
    \item[??] BES III
%\end{itemize}
\end{itemize}

% Norman Christ
\subsection{$K_L$-$K_S$ mass difference}
\begin{itemize}
    \item[Motivation.] This is a precisely measured quantity that is highly sensitivity to new physics 
    at an energy scale of $10^4$ TeV.
    \item[Long term goal.] Calculate $K_L - K_S$ mass difference with
    a precision exceeding the current experimental value of $3.484
    \pm 0.007 \times  10^{-12}$ MeV.
    \item[Method.] Well understood lattice QCD technique with no
    recognized limitation to the control of all systematic
    errors. GIM cancellation is essential and treatment of the
    charmed quark using QCD perturbation theory introduces 36\%
    errors. The lattice calculation is made difficult by the
    requirement of a lattice spacing smaller than the charm quark
    Compton wave length and a volume sufficiently large to accurate
    treat physical mass pions. 
\item[Timeline:]
    \item[2010] Accurate KTeV result~\cite{KTeV:2010sng}.
    \item[2010-2014] Lattice QCD method devised and first results 
    obtained.~\cite{Christ:2012se, Bai:2014cva}.
    \item[2014-2021] First results with physical masses and $1/a = 2.38$ MeV 
    obtained, presence of large discretization errors recognized~\cite{Bai:2018lrm, 
    Wang:2021twm}.
    \item[2021-2023] Discretization errors studied and evidence of $a^2$ scaling 
    found.
    \item[2023-2024] Calculation of the long-distance contribution to
    $\varepsilon_K$ with $1/a = 2.38$ MeV will include extension of
    earlier $\Delta M_K$, calculation, increasing statistics.
    \item[2024-2026] Calculation of $\Delta M_K$ with $1/a = 2.8$ GeV
    giving continuum limit result with $20$\% errors. 
    \item[2027-2030] Move to $2+1+1$ flavors and $1/a = 3.0$, $4.0$
    GeV and possibly 5 GeV with $10$\% errors. 
\end{itemize}

% Norman Christ
\subsection{Long distance contribution to $\epsilon_K$}
\begin{itemize}
    \item[Motivation.] The indirect CP violation parameter $\epsilon_K$ is precisely 
    measured and sensitive to BSM sources of CP violation at high energies. Its 
    accurate standard model (SM) prediction provides a critical test of the KM 
    theory of CP violation.  While 95\% of the SM prediction comes from short 
    distance and is computed to an accuracy approaching 1\%, there is a 5\% 
    contribution from long distances on the scale of the charm quark Compton 
    wavelength and larger that requires a lattice QCD calculation.
    \item[Long term goal.] Calculate the long-distance contributions to $\epsilon_K$ 
    to 10\% precision, sufficient to match or exceed the current experimental 
    $\epsilon_K$ precision: $|\epsilon_K|=2.228\pm 0.011 \times 10^{-3}$~\cite{ParticleDataGroup:2022pth}.
    \item[Method.] Well understood lattice QCD technique with no recognized 
    limitation to the control of all systematic errors. GIM cancellation is essential 
    but a logarithmic divergence in the effective long-distance theory requires the 
    perturbative calculation of a low energy constant which becomes systematically 
    more accurate as the matching scale is increased above the charm quark mass.
\item[Timeline:]
    \item[2000] KTeV and NA48 measurements of $\epsilon'$.
    \item[2014-2017] Lattice QCD method devised and first results 
    obtained~\cite{Christ:2015phf}. 
    \item[2023-2024] First calculation with physical quark masses but relatively 
    coarse $1/a=2.38$ GeV lattice spacing (combined with an extension of the
    $\Delta M_K$ calculation with increased statistics).
    \item[2024-2026] Calculation including $\Delta M_K$ with $1/a = 2.8$ GeV 
    giving continuum limit result with $20$\% errors. 
    \item[2027-2030] Move to $2+1+1$ flavors and $1/a = 3.0$, $4.0$
    GeV and possibly 5 GeV with $10$\% errors. 
\end{itemize}



% Norman Christ
\subsection{Direct CP-violation in kaon decays ($\epsilon'$)}
\begin{itemize}
    \item[Motivation.] Highly sensitive to new sources of CP that may explain the matter/antimatter asymmetry in the observable Universe. Experimental result with ${\cal O}(10\%)$ precision already available.
    \item[Long term goal.] Calculate $\epsilon'$ to a precision exceeding the current 10\% experimental value.        
    \item[Method.] Lattice calculation of $K\to\pi\pi$ matrix elements using 3-flavor weak effective theory in finite-volume allows extraction of infinite-volume amplitudes, $A_2$ \& $A_0$, through Lellouch-L\"uscher methods. Antiperiodic ($A_2$) and novel ``G-parity'' ($A_0$) boundary conditions (BCs) ensure dominance in signal of physical, energy-conserving decay. Non-perturbative renormalization at high energies minimizes systematic error in matching to perturbation theory describing weak interaction physics. Vacuum contributions to $A_0$ require large statistics to adequately resolve. Future calculations aim to reduce systematic errors by incorporating electromagnetism and isospin-breaking effects, and an active charm quark.

 % Norman Christ
\item[Timeline:]
%\begin{itemize}
    \item[1999] First definitive observations of kaon direct CP-violation at KTeV (FermiLab) and NA48 (CERN).
    \item[2001] Final NA48 experimental result published~\cite{NA48:2001bct}.
    \item[1985-2002] Early lattice calculations using quenched QCD and chiral perturbation theory (ChPT) obtained results with large, uncontrolled systematic errors.
    \item[2004] Lattice calculation with dynamical QCD and ChPT~\cite{Li:2008kc}. Large systematic errors due to ChPT discouraged continued usage of this approach.
    \item[2011] Final KTeV experimental result published~\cite{KTeV:2010sng}. Combining experimental results gives current, best determination ${\rm Re}(\epsilon'/\epsilon)=16.6(2.3)\times 10^{-4}$.
    \item[2011] Development of lattice approach for computing $K\to\pi\pi$ decays directly~\cite{Blum:2011pu}.
    \item[2012] First {\it ab initio}, physical calculation of $A_2$~\cite{Blum:2011ng,Blum:2012uk}.
    \item[2015] First continuum-limit calculation of $A_2$~\cite{Blum:2015ywa}. First {\it ab initio}, physical calculation of $A_0$ and $\epsilon'$ using G-parity BCs~\cite{RBC:2015gro}.
    \item[2020] Improved calculation of $A_0$ and $\epsilon'$ with G-parity BCs obtained substantially smaller statistical errors and better control over systematic errors~\cite{RBC:2020kdj}. This, latest result agrees with experiment but has ${\sim}4\times$ the total error.
    \item[2020-2024] Repeat calculation of $A_0$ on finer lattices to reduce/remove ${\cal O}(12\%)$ discretization systematic error.
    \item[2020-2026] Develop new methods to incorporate electromagnetism and isospin-breaking effects, reducing dominant ${\cal O}(23\%)$ systematic error.
    \item[2024-2026] Develop approach to incorporating active charm quarks with controlled discretization errors, reducing an ${\cal O}(12\%)$ systematic error.
    \item[2026-2031] Improved measurements employing new approaches, aiming to match/exceed experimental precision by end of Snowmass '21 period. Potential discovery of tension between experiment and Standard Model may lead to new generation of kaon experiments.   
%\end{itemize}
\end{itemize}

% Norman Christ/Tom Blum
\subsection{Direct $CP$ violation in $K\to\pi\pi$ decays using periodic boundary conditions}
\begin{itemize}
    \item[Motivation.] Precise SM prediction of $\varepsilon'$, the measure of direct $CP$ violation in $K\to\pi\pi$ decay, which is very sensitive to new physics. 
    \item[Long term goal.] Calculate Re($\varepsilon'/\varepsilon$) with
      a precision exceeding the current experimental value of $1.66(23)\times10^{-3}$, which roughly corresponds to 10\% precision of $A_0$, the decay amplitude with isospin-0 final state.
    \item[Method.] Well understood lattice QCD technique with no
      recognized limitation to the control of all systematic
      errors. The $K\to\pi\pi$ matrix elements with the physical kinematics are to be
      extracted by using the well developed Generalized eigenvalue problem (GEVP)
      method~\cite{Luscher:1990ck,Bulava:2011yz}.  The Lellouch-L\"uscher
      formalism~\cite{Lellouch:2000pv} has been used to relate finite- and
      infinite-volume two-pion states in isospin-symmetric calculations but
      needs to be extended for introduction of electromagnetism, which is
      expected to give a significant ($O(20\%)$) impact on $\varepsilon'$
      but has been absent from earlier lattice calculations.  Effects of the
      charm quark and naive discretization effects, which are also significantly
      uncertain for achieving our precision goal at this point, are to be both
      reduced by calculating on finer lattices.
\item[Timeline:]
%\begin{itemize}
    \item[1999] First definitive observations of kaon direct CP-violation at KTeV (FermiLab) and NA48 (CERN).
    \item[2001] Final NA48 experimental result published~\cite{NA48:2001bct}.
    \item[1985-2002] Early lattice calculations using quenched QCD and chiral perturbation theory (ChPT) obtained results with large, uncontrolled systematic errors.
    \item[2004] Lattice calculation with dynamical QCD and ChPT~\cite{Li:2008kc}. Large systematic errors due to ChPT discouraged continued usage of this approach.
    \item[2011] Final KTeV experimental result published~\cite{KTeV:2010sng}. Combining experimental results gives current, best determination ${\rm Re}(\varepsilon'/\varepsilon)=16.6(2.3)\times 10^{-4}$.
    \item[2011] Development of lattice approach for computing $K\to\pi\pi$ decays directly~\cite{Blum:2011pu}.
    \item[2012] First {\it ab initio}, physical calculation of $A_2$~\cite{Blum:2011ng,Blum:2012uk}.
    \item[2015] First continuum-limit calculation of $A_2$~\cite{Blum:2015ywa}. First {\it ab initio}, physical calculation of $A_0$ and $\varepsilon'$ using G-parity BCs~\cite{RBC:2015gro}.
    \item[2020] Improved calculation of $A_0$ and $\varepsilon'$ with G-parity BCs obtained substantially smaller statistical errors and better control over systematic errors~\cite{RBC:2020kdj}. This, latest result agrees with experiment but has ${\sim}4\times$ the total error.
    \item[2023] Excited two-pion state that corresponds to the final state of energy-conserving $K\to\pi\pi$ decay with periodic boundary conditions (PBC) successfully extracted~\cite{Blum:2023vhp}.  First PBC result for $\varepsilon'$ with $a^{-1}=1.02$~GeV going to be released (paper almost ready).  It agrees with experiment and the G-parity result, but the precision is worse due to fewer configurations and coarser lattice used.
    \item[2022-2024] Continue PBC calculations with more configurations and on a finer lattice of $a^{-1}=1.38$~GeV to obtain a more precise result with reduced statistical and discretization errors.
    \item[2020-2026] Develop new methods to incorporate electromagnetism and isospin-breaking effects, reducing dominant ${\cal O}(20\%)$ systematic error.  (See item below.)
    \item[2024-2026] Repeat calculations of $A_0$ on even finer lattices $a^{-1}=1.7, 2.3, 2.7$~GeV. 
    \item[2026-2030] Improved measurements including isospin-breaking and electromagnetic corrections, aiming to match/exceed experimental precision. Potential discovery of a tension between experiment and the SM may prompt a new generation of $\varepsilon'$ experiments. 
\end{itemize}
% Norman Christ
\subsection{Contribution of E\&M and strong isospin breaking to $\epsilon'$}
\begin{itemize}
    \item[Motivation.] Because of the $\Delta I=1/2$ rule, the isospin breaking corrections to the direct CP violating parameter $\epsilon'$ are enhanced by a factor of 20, increasing their usual 1\% scale to potentially 20\% and making the uncertainty in these corrections one of the dominant errors in the current standard model prediction of $\epsilon'$.
    \item[Long term goal.] Calculate the isospin breaking corrections to $\epsilon'$ to 10\% accuracy removing this as a source of error in the prediction of $\epsilon'$.
    \item[Method.] The addition of electromagnetism adds substantial barriers to the already challenging lattice calculation of $\epsilon'$:  i) The $I=0$ and $I=2$ $\pi\pi$ final states are mixed and the three-particle $\pi\pi\gamma$ state introduced making this a complicated, multi-channel decay. ii)  Standard lattice formulations of QED have substantial finite-volume errors while the successful lattice methods to compute $\epsilon'$ require that the calculation be performed in a volume of limited size.  iii) Adding photons substantially increases the complexity of the quantities being computed, likely requiring the use of computer generated code.
\item[Timeline:]
    \item[2000] KTeV and NA48 measurements of $\epsilon'$
    \item[2017] Non-relativistic method to compute the Coulomb contribution to $\epsilon'$ devised~\cite{Christ:2017pze}.
    \item[2019-2021] Generalize this non-relativisitic method to one that is fully relativistic~\cite{Christ:2021guf}.
    \item[2023-2024] Explore two strategies:  i) Use lattice methods to evaluate the low energy constants needed in a chiral perturbation theory calculation~\cite{Cirigliano:2019cpi} and ii) Extend the Coulomb-gauge approach to include transverse radiation.  Begin a calculation based on the most ;promising method.
    \item[2024-2027] Calculation of isospin breaking contributions to $\epsilon'$ to 30\% accuracy.
    \item[2027-2030] Calculation of isospin breaking contributions to $\epsilon'$ to 10\% accuracy.
\end{itemize}

% Norman Christ
\subsection{Two photon exchange contribution to $K_L\to\mu^+\mu^-$}
\begin{itemize}
    \item[Motivation.] The $\Delta S=1$ neutral-current rare decay $K_L\to\mu^+\mu^-$ 
    offers an order $G_F^2$ test of the standard model in a process involving the 
    exchange of two $W$ bosons or a $W$ and a $Z$ boson.  However, a background, 
    two-photon exchange process of order $\alpha_{\mathrm{EM}}^2G_F$ contributes 
    at a similar strength to this decay and therefore must be computed if a standard 
    model prediction is to be compared with experiment.
    \item[Long term goal.] Calculate this two-photon exchange contribution to 5\% 
    accuracy so that the standard model prediction for the important $G_F^2$ process 
    can be compared to experiment at the 5-10\% precision comparable to the accuracy 
    allowed by the current experimental branching ratio for $K_L\to\mu^+\mu^-$: 
    BR$(K_L\to\mu^+\mu^-) = 6.84\pm0.11\times 10^{-9}$.
    \item[Method.]  This process is similar to the hadronic light-by-light scattering 
    contribution to the muon.  However, new methods are needed because the 
    real-time, complex decay amplitude cannot be directly evaluated in Euclidean 
    space.  However, a well-controlled lattice QCD formulation of this calculation 
    has been developed in which the largest uncontrolled error arises from the 
    $\pi\pi\gamma$ intermedate state whose contribution is expected to be no 
    more than a few percent.
\item[Timeline:]
    \item[2000] Accurate measurement of $K_L\to\mu^+\mu^-$ branching 
    ratio~\cite{E871:2000wvm}.
    \item[2018-2019] Lattice QCD method devised~\cite{Christ:2020bzb} and 
    first results obtained for the simpler $\pi^0\to e^+ e^-$ 
    decay~\cite{Christ:2020dae, Christ:2022rho}.
    \item[2019-2022] Calculation extended to the $\Delta S=1$ process 
    $K_L\to\gamma\gamma$ and first results obtained on a single gauge 
    ensemble including only connected graphs with physical quark masses 
    but for a relatively large lattice spacing.~\cite{Zhao:2022pbs, Zhao:2022njd}
    \item[2022-2024] First calculation of the two-photon exchange contribution 
    to $K_L\to\mu^+ \mu^-$ on a single gauge ensemble including only connected 
    graphs with physical quark masses but for a relatively large lattice spacing.
    \item[2025-2028] Extend the calculation to multiple lattice spacings and include 
    disconnected graphs.  Achieve 10\% precision.
    \item[2028-2031] Employ increased statistics and improved methods to 
    achieve the targeted 5\% precision.
 \end{itemize}
 
\section{Hadron spectroscopy}\label{sec:hadspec}

% Robert Edwards
\subsection{Exotic light quark mesons}
\begin{itemize}
    \item[Motivation.] Observation of putative exotic $J^{PC}$ bosonic
      states. Focus of GlueX@JLab, CLAS12@JLab, COMPASS@LHC. Kaon PID upgrade of GlueX and recommissioning starting 2020. Phase II including JLab Eta Factory underway through 2025. Possible Phase III running.
    \item[Long term goal.] Determine the mass and decay modes of
      putative light quark hybrid and exotic mesons. Extract resonance
      parameters of meson and baryon spectrum. Guide experimental
      searches with predictions of decay couplings.
    \item[Method.] Resonance spectrum extracted from scattering
      amplitudes computed from finite-volume energy spectrum. Cost
      driven by annihilation quark lines computed on many
      time-slices and computation of hadronic two-point functions
      featuring a large multi-hadron operator basis utilizing the variational method. 
\item[Timeline:]
%\begin{itemize}
    \item[2013] Isovector and isoscalar light quark meson and baryon
      spectrum computed with restriction to single particle basis~\cite{Dudek:2013yja}. Results featured in PDG.
    \item[2015] First computation of resonance parameters from coupled-channel scattering
      amplitudes of $\pi\pi/\bar{K}K$ \cite{Wilson:2015dqa}.
    \item[2019] Phenomenological extraction of $\pi_1$ resonance from 
      parameters from partial waves of $\eta(')\pi$ measured by COMPASS~\cite{JPAC:2018zyd}.
    \item[2022] First determination of full three-body relativistic scattering amplitude~\cite{Hansen:2020otl}
    \item[2022] Prediction for light-quark isovector $J^{PC}=1^{-+}$ published~\cite{Woss:2020ayi}.
    \item[2023] Upgrade of GlueX forward calorimeter.
    \item[2024] Putative light-quark hybrid meson multiplet resonance parameters.
    \item[2025] GlueX Phase II completed. Begin analysis of runs I and II. 
    \item[2027] GlueX results based on combined data through run II.
%\end{itemize}
\end{itemize}



% Robert Edwards
\subsection{Internal structure of resonant states}
\begin{itemize}
    \item[Motivation.] The internal structure of QCD resonant states
      is poorly understood. Their study is a focus of GlueX@JLab, CLAS12@JLab, LHCb@CERN.
    \item[Long term goal.] Use electromagnetic probes to reveal the internal structure of resonant
      states, revealing potential multi-particle
      configurations through computation of physically relevant
      processes, like transition form-factors, or more complicated
      structures such as partonic content.
    \item[Method.] Combines QCD spectrscopy program and hadronic
      structure programs. Infinite volume current matrix elements
      extracted from scattering amplitudes.
\item[Timeline:]
%\begin{itemize}
    \item[2015] First computation of $\gamma\pi\rightarrow\pi\pi$ form-factor~\cite{Briceno:2015dca}. 
    \item[2022] First computation of $\gamma K\rightarrow K\pi$
      form-factor~\cite{Radhakrishnan:2022ubg} featuring multi-partial
      wave formfactors.
    \item[2024] Multi-channel study of charge radius for $a_0(980)$.
    \item[2024] Multi-channel nucleon transition form-factors of
      vector and axial currents.
    \item[2025] Radiative transition form-factors of exotic isovector
      $\pi_1$.
    \item[2026] GlueX electromagnetic branching fraction rates for
      exotic meson states
%\end{itemize}
\end{itemize}




\section{Hadron structure}\label{sec:hadstruct}

\subsection{Electromagnetic Form Factors}\label{sec:vecff}

% Rajan Gupta
\subsection{Nucleon charges}\label{sec:nuccharges}
\begin{itemize}
    \item[Motivation.] Nucleon charges $g_{A,S,T}$ arise in many
      low-energy description of nucleons. Flavor diagonal axial
      charges give the contributions of each quark flavor to the
      nucleon spin; tensor charges give the contribution of quark EDM
      to the nucleon EDM; and scalar charges give the pion-nucleon sigma term and strangeness content of the nucleon. 
      Isovector charges give the axial charge $g_A$, a fundamental low-energy constant. The scalar and tensor 
      charges probe novel scalar and tensor interactions at the TeV scale and the tensor charge is also 
      measured in transversity measurements at JLAB. From the matrix elements of 1-link operators, we extract 
      the first moment of distributions, namely momentum fraction, and helicity and transversity moments. These 
      results compare favorably with their extractions from PDFs.
    \item[Long term goal.] To calculate each of these with sub-percent accuracy. 
    \item[Method.] All these require the calculation of connected and disconnected three-point
      functions comprising the insertion of appropriate quark bilinear operators between 
      nucleon source and sink operators. During analysis, remove the
      excited state contributions from these correlation functions to
      get ground state matrix elements that are then decomposed into
      the desired charges. The calculation is repeated at
      multiple values of lattice spacing and pion mass and the results
      extrapolated to the physical point.
\item[Timeline:]
    \item[2011--2019] Lattice results reach the robustness standard to be reviewed by FLAG and average values publushed.
    \item[2019-2023] Steady improvement in precision to reach few  percent accuracy and get control over excited states. 
      Possible resolution of the discrepancy between lattice estimate of the the pion-nucleon sigma term between lattice QCD and 
      dispersive analysis published by the PNDME collaboration. 
    \item[2023-2025] Continue to improve accuracy. 
    \item[2030] Much higher precision analysis of nuetron decay that
      would push the search for novel scalar and tensor interactions to the $10^{-4}$ level. 
\end{itemize}

% Rajan Gupta
\subsubsection{Nucleon electromagnetic form factors}
\begin{itemize}
    \item[Motivation.] The electromagnetic form factors of the nucleons are well measured experimentally from electron 
      scattering off nuclei.  There is still uncertainty in the calculation of the charge radius of the proton, i.e., 
      a difference from scattering experiments versus muonic hydrogen.  These data are far more precise compared to 
      lattice results and for the foreseeable future will provide a benchmark against which to compare lattice reults to 
      and validate lattice methodology. 
    \item[Long term goal.] To determine the electromagnetic form factors of the nucleons to match experimental precision. 
    \item[Method.] First calculate three point functions comprising the insertion of the electromagnetic current
      with all allowed lattice momentum insertion between the nucleon source and sink operators. During analysis, remove 
      the excited state contributions from these correlation functions to get ground state matrix elements that are 
      then decomposed into the desired form factors. The calculation is repeated at multiple values of lattice spacing and 
      pion mass and the results extrapolated to the physical point. 
\item[Timeline:]
    \item[2011--2021] Lattice results showed deviations from the Kelly parameterization of experimental data. 
    \item[2021] NME results agree with Kelly parameterization of experimental data.
    \item[2021-2023] Calculations by various groups still show an uncertainty band of $\approx 10\%$.
    \item[2023-2025] NME and PNDME collaborations to reduce the uncertainty band to 3--5\%
    \item[2030] EIC expected to start taking data.
\end{itemize}

% Sergey Syritsyn
\subsubsection{Calculation of high-momentum nucleon form factors}
\begin{itemize}
    \item[Motivation.] Any data on nucleon form factors in the range of momentum transfer 5-20 GeV$^2$ will be important for understanding the transition between perturbative and nonperturbative QCD. Such data will also be illuminating for qualitative pictures of nucleons in terms of gluon-dressed constituents or diquarks.
    \item[Long term goal.] Ab initio calculation of such form factors using QCD on a lattice, done with fully controlled systematic errors, is necessary to fill the gap between theory and ongoing experiments. Comparison to experimental results will also reveal and help address any systematic uncertainties in calculations with high-momentum hadron states on a lattice such as calculations of PDFs/GPDs using LaMET and similar approaches.
    \item[Method.] We employ standard lattice techniques together with some methods (e.g., momentum-boosted states) to improve overlap with large-momentum nucleon ground state. We use Breit frame kinematics to reach maximal possible momentum transfer.
\item[Timeline:]
    \item[2016-2023] We have performed calculations of electric and magnetic form factors at three values of lattice spacing and pion masses down to 170 MeV. We have found qualitative agreement in the dependence of the proton form factor ratios $G_{Ep}/G_{Mp}$ and $F_{1p}/F_{2p}$. Some agreement is also evident in the neutron case for $G_{En}/G_{Mn}$. While this agreement is reassuring, the values of the form factors themselves overshoot experiment by factor 2-2.5 for momenta $Q^2$ above 4 GeV$^2$. This is probably due to large contributions of the excited states, since we observe very little dependence on the lattice spacing and the pion mass.
    \item[2023-onward] We plan to increase the statistical precision and extend the range of nucleon correlators in order to improve the control over excited states. In addition, we plan to study the so-called disconnected diagrams (contributions to the isoscalar $(p+n)$ channel) and the leading-order $O(a)$ discretization corrections.
    \item[Relevant experiments] have been performed at CEBAF (JLab@12 GeV) to measure electromagnetic form factors up to 18 GeV$^2$. Results for the magnetic form factor of the proton have been published.
\end{itemize}

\subsection{Axial Form Factors}\label{sec:axff}

% Rajan Gupta
\subsubsection{NME, PNDME Collaborations}
\begin{itemize}
    \item[Motivation.] The axial form factors of the nucleons are not well measured experimentally and are
      needed to few percent accuracy to calculate the neutrino-nucleus cross-section for experiments such as DUNE, T2K, etc., to 
      reach their science goals. Large scale simulations of lattice QCD are the current 
      best method to determine these directly from QCD and the methodology for the 
      calculations is mature. 
    \item[Long term goal.] To determine the axial form factors of the nucleons to a few percent 
    \item[Method.] First calculate three point functions comprising the insertion of the axial vector and pseudoscalar currents 
      with allowed lattice  momentum insertion between the nucleon source and sink operators. During analysis, remove 
      the excited state contributions from these correlation functions to get ground state matrix elements that are 
      then decomposed into the desired form factors. The calculation is repeated at multiple values of lattice spacing and 
      pion mass and the results extrapolated to the physical point. 
\item[Timeline:]
    \item[2017] Demonstration that the standard method of analysis gives form factors that do not satisfy PCAC.
    \item[2019] Cause of failure identified to be the contributions of multihadron ($N\pi, N\pi\pi, \ldots$) states 
      and a data driven method to remove them identified.
    \item[2021-2023] Calculations done by a number of lattice collaborations that agree within $1\sigma$ and give results
      with an uncertainty band of $\approx 10\%$.
    \item[2023-2025] NME and PNDME collaborations to reduce the uncertainty band to 3--5\%
    \item[2030] DUNE expected to start taking data.
\end{itemize}

\subsubsection{Fermilab Lattice Collaboration}
\begin{itemize}
   \item[Motivation:] Explore feasibility of nucleon-matrix-element calculations with staggered valence quarks, motivated by the neutrino-scattering experiments listed in the previous item.
   \item[Long term goal.] Axial charge $g_A$ and axial form factor with the HISQ action for valence quarks, using MILC's $2+1+1$-flavor ensembles.
   \item[Method:] Staggered fermions have a more complicated symmetry structure than Wilson-like quarks (including domain-wall, twisted-mass, clover, \ldots), making nucleon operators more complicated to construct.  While we have solved this problem for the needed two- and three-point calculations, the approach does not turn out to enjoy the computational savings that staggered fermions provide for mesons.
    \item[Computing:] Work on ensembles with $a\approx0.15$ and 0.12~fm were carried out on USQCD resources (FNAL~LQ1, BNL~Sklake, and BNL~KNL), while work on ensembles with $a\approx0.09$ and 0.06~fm were carried out on LCF resources (NERSC Cori, XSEDE Stampede2, ALCF Theta).
\item[Timeline:]
   \item[2019] Calculation of the nucleon mass at three spacings with physical quark mass~\cite{Lin:2019pia}; at the time (and possibly still), this was the most precise calculation of the nucleon mass.
   \item[2021] Demonstration calculation of the axial charge $g_A$ on the physical-quark-mass HISQ ensemble with $a\approx0.12$~fm~\cite{Lin:2020wko}.
   \item[2023] Calculation of $g_A$ at four spacings ($a\approx0.15$--0.06~fm) with physical quark mass, with update to the nucleon mass.
   \item[2023] Demonstration calculation of the axial charge $g_A$ on the physical-quark-mass HISQ ensemble with $a\approx0.12$~fm~\cite{Lin:2020wko}.
\item[Outlook:] The all-HISQ approach to baryons seems not to be cost-effective enough to justify the overhead in coding and analysis.
\end{itemize}



\subsection{Nuclean Electric Dipole Moment (nEDM)}\label{sec:nEDM}

% Rajan Gupta
\subsubsection{Contributions of CP violating operators to nucleon electric dipole moment}
\begin{itemize}
    \item[Motivation.] The CP violation in the CKM quark mixing is too
      small to generate the observed baryon asymmetry in the observed
      universe.  BSM models have new sources of CP violation, each of
      which contribute to nEDM. These interactions and their couplings
      in BSM theories are, using tools of effective field theories,
      written in terms of low energy operators composed of quark and
      gluon fields. The matrix elements of these effective operators
      are only known within a factor of 10 uncertainty.  Lattice QCD
      provides the best method to determine them with control over all
      systematics. Knowing these matrix elements, one can use the bound
      [eventually value] of nEDM to constrain the CP violating
      couplings in various BSM theories.  These viable BSM models can
      then be analyzed to determine if they are further consistent
      with the generation of the observed baryon asymmetry.
    \item[Long term goal.] To calculate the matrix elements of all low
      energy effective operators of dimension six 
      and smaller that contribute to the nucleon electric dipole moment. 
    \item[Method.] This requires calculating three- and four- point
      functions comprising the insertion of the electromagnetic
      current with finite CP violating operators with finite momentum
      insertion between the nucleon source and sink operators to be
      able to take the zero momentum limit.  During analysis, remove
      the excited state contributions from these correlation functions
      to get ground state matrix elements that are then decomposed
      into the desired form factors of which $F_3$ is the desired CP
      violating one from which $d_n = F_3(0)/2 M_N \epsilon$, where
      $\epsilon $ is the coupling strength of the operator. Repeat the
      calculation at multiple values of lattice spacing and pion mass
      and extrapolate the results to the physical point.
\item[Timeline:]
    \item[Pre 2017] The first lattice calculations set up the
      framework but did not include a phase in the definition of the
      nucleon spinor, which lead to wrong results.  
    \item[2017-2023] Focus on error reduction in the extraction of
      $F_3$ parameterization of experimental data. Understand and remove 
      the contributions of multihadron ($N\pi,
      N\pi\pi, \ldots$) states. 
    \item[2021-2023] Calculations from a number of collaboration for
      the Theta term with different levels of control over signal and
      systematics.  
    \item[2023-2025] To get a robust signal at near physical quark masses. 
    \item[2030] Expect a factor of 10 reduction in the upper bound on
      neutron EDM and mabe a value.  
\end{itemize}

% Sergey Syritsyn
\subsubsection{Quark-gluon CP violating contributions}
\begin{itemize}
    \item[Motivation.] Neutron and proton electric dipole moments
      (EDMs) are the most precise probes of CP violation in QCD and
      beyond-the SM physics. The latter is important for finding
      plausible scenarios of baryogenesis and understanding the origin
      of the nuclear matter. Knowing the magnitude of neutron EDM
      induced by different sources of CP violation is crucial to using
      experimental bounds to constrain new types of CP-violating
      interactions. 
    \item[Long term goal] is finding contributions to nucleon EDMs of
      effective quark-gluon CPv operators that may be induced by new
      physics. These effective CPv interactions are quark EDMs, quark
      and gluon chromo-EDMs, QCD theta-term, and 4-quark
      interactions. 
    \item[Method.] We use the background-field method in which the
      energy (mass) of a nucleon is shifted due to induced EDM. This
      method may have comparative advantage compared to computing
      electric dipole form factors that have to be extrapolated to the
      forward limit. In addition, it may be simpler to implement in
      the case of four-quark operators. 
\item[Timeline:]
    \item[2016-2023] We have successfully demonstrated that the
      background field method is at least comparable in precision to
      the traditional method of computing neutron EDM induced by the
      QCD theta-term.We have implemented sophisticated sampling
      techniques combining variable precision and low-lying eigenmode
      approximation of the Dirac equation. We have performed
      calculation of the neutron EDM using chirally-symmetric
      fermions. 
    \item[2023-onward] We plan to extend our calculations of
      theta-QCD-induced neutron EDM to the physical point, where the
      low-eigenmode approximation is expected to be even more
      efficient. We also plan to start calculations involving 4-quark
      CPv operators. 
    \item[Experiments] are expected to improve bounds on neutron EDMs
      within the current decade. It is particularly important to
      improve our theoretical knowledge of neutron EDM contributions
      in the same time frame in order to constrain proposed models of
      novel CP-violating interactions and baryogenesis scenarios. 
\end{itemize}



% David Richards
\subsection{Internal Structure of Mesons and Nucleons}\label{sec:meshadstruct}
\begin{itemize}
   \item[Motivation.] A quantitative understanding of the internal
     structure of hadrons, including the distribution of charge, spin
     and mass, in terms of the quarks and gluons of QCD is a central
     goal of nuclear physics, and key to high-energy physics through
     exposing the Standard Model and Beyond-the-Standard Model
     interactions of the quarks and gluons within them.  Flagship
     experiments at JLab\@12 GeV focused on isovector GPDs, and
     understanding the gluonic contributions to hadron structure a
     emblematic problem at the future EIC.
   \item[Long term goal.] First principles calculations of the key
     measures of one-dimensional and three-dimensional hadron
     structure encapsulated in the Parton Distribution Functions (PDFs),
     Generalized Parton Distribution Functions (GPDs) and
     Transverse-Momentum-Dependent Functions (TMDs).  Performing
     computations of a precision that can both confront experiment,
     and complement experiment in global analysis.
   \item[Method.] Computation of the three-point matrix elements of quark- and
     gluonic operators, and their analysis to provide
     Bjorken-$x$-dependent distributions within the
     short-distance-factorization/pseudo-PDF framework.  Exploitation
     of the ``distillation'' both to fully sample lattices, and
     provide control over excited-state contributions.
\item[Timeline:]
   \item[2017] First calculation of the nucleon unpolarized PDF within
     the pseudo-PDF framework~\cite{Orginos:2017kos}.
   \item[2020] Calculation of unpolarized nucleon PDF at close-to-physical
     quark masses~\cite{Joo:2020spy}.
   \item[2021] Calculation of the Unpolarized Gluon Distribution in
     the Nucleon~\cite{HadStruc:2021wmh}.
   \item[2022] Calculation of the gluon helicity distribution in
     the nucleon~\cite{HadStruc:2022yaw}
   \item[2022]Ccombined global analysis of lattice QCD and
     experimental data to compute the Pion PDF~\cite{JeffersonLabAngularMomentumJAM:2022aix}.
   \item[2024] Calculation of the isosinglet unpolarized and
     polarized nucleon PDF.
   \item[2024] First Calculation of the nucleon Generalized Parton
     Distributions within the pseudo-PDF framework.
   \item[2027] E12-06-119 ``Deeply Virtual Compton Scattering with
     CLAS at 11 GeV''
   \item[2027] SOLID detector at Jefferson Lab
     \item[2030] Electron-Ion Collider at BNL.
\end{itemize}


% Michael Engelhardt    
\subsection{Transverse momentum-dependent parton distribution
(TMD) observables.}
\begin{itemize}
\item[Motivation.] TMDs constitute one of the pillars of the description
of hadron structure; they encode the three-dimensional distribution of
momenta among the partons inside a hadron. They enter the cross sections
of processes such as semi-inclusive deep inelastic scattering (SIDIS) and
the Drell-Yan process. Their determination constitutes a focus of both the
current JLab 12 Gev program as well as the experimental program at the EIC.
\item[Long term goal.] Calculation of TMD observables to
complement and augment phenomenological analyses of experimental data,
and to help guide follow-on experimental campaigns.
\item[Method.] Lattice QCD evaluation of hadronic matrix elements of
bilocal operators containing staple-shaped gauge connections. Extraction
of invariant amplitudes that determine ratios of Fourier-transformed TMDs.
\item[Timeline:]
\item[2011] First calculation of Sivers and Boer-Mulders observables.
\item[2015] Extrapolation of a TMD observable to the physical, infinite
rapidity difference limit.
\item[2017] Investigation of universality of lattice TMD observables.
\item[2019] Evaluation of TMD observables at the physical pion mass.
\item[2022] Completion of leading-twist set of nucleon TMD observables by
including longitudinal nucleon polarization.
\item[2024] Evaluation of gauge connection structures beyond staple shape,
relevant for processes beyond SIDIS and Drell-Yan.
\item[2024] Evaluation of longitudinal momentum fraction $x$-dependence
of the Sivers shift.
\item[2027] Inclusion of Lattice TMD observables with controlled systematic
uncertainties in global analyses of experimental data.
\end{itemize}


% Michael Wagman
\subsection{Collins-Soper kernel for TMD evolution}
\begin{itemize}
    \item[Motivation.] The Collins-Soper kernel relates the
      transverse-momentum-dependent distributions (TMDs) of hadrons at
      different energy scales, which is a crucial non-perturbative
      input for global analyses to understand the 3D structure of the
      proton. 
    \item[Long term goal.] Calculate the Collins-Soper kernel at
      physical pion mass with a $\sim 10\%$ precision, which can be
      compared to the experimental results from the Electron-Ion
      Collider. 
    \item[Method.] Lattice calculation and renormalization of the
      matrix elements of quasi-TMD correlators in a highly boosted
      hadron state. The quasi-TMD correlators are defined from quark
      and gluon bilinear operators with a staple-shaped Wilson line
      extending to the longitudinal direction as the hadron
      momentum. At large hadron momentum, the quasi-TMD can be
      factorized into the physical TMD, and the anomalous dimension of
      their momentum evolution can be perturbatively matched onto the
      Collins-Soper kernel. Apart from the forward matrix elements for
      quasi-TMDs, we can also use the TMD wave function, which is a
      vacuum to hadron amplitude of the quasi-TMD correlator, to
      extract the Collins-Soper kernel. 
\item[Timeline:]
%\begin{itemize}
    \item[2019] Study of the renormalization and mixing of the
      staple-shaped quasi-TMD correlator on the
      lattice~\cite{Shanahan:2019zcq}. 
    \item[2020] First exploratory calculation of the quark
      Collins-Soper kernel on a quenched lattice ensemble from the
      quasi-TMDs~\cite{Shanahan:2020zxr}. 
    \item[2021] First lattice calculation of the quark Collins-Soper
      kernel with dynamical fermions at unphysical pion mass of around
      500 MeV from the quasi-TMDs~\cite{Shanahan:2021tst}. 
    \item[2022-2023] Ongoing lattice calculation of the quark
      Collins-Soper kernel at the physical point from the quasi-TMD
      wave functions. 
    \item[2023-2025] Improving the precision of the quark
      Collins-Soper kernel to less than $10\%$ with physical
      extrapolations and higher hadron momenta. 
    \item[2023-2026] Develop and implement the method to calculate the gluon Collins-Soper kernel.
    \item[2023-2026] Lattice calculation of the quark TMD soft factor.
%\end{itemize}
\end{itemize}

% Michael Engelhardt    
\subsection{Direct evaluation of parton orbital angular momentum (OAM) in
the proton}
\begin{itemize}
\item[Motivation.] The proton spin puzzle is a touchstone of our
understanding of proton structure. Compared to the parton spin
contributions, the parton OAM contributions are less straightforward
to access directly, requiring the evaluation of either generalized
transverse momentum-dependent parton distribution (GTMD) observables
or twist-three generalized parton distribution (GPD) observables.
Lattice QCD provides an avenue to obtain these observables.
\item[Long term goal.] Evaluation of GTMD and twist-three GPD observables
to determine parton OAM in the proton.
\item[Method.] Lattice QCD evaluation of hadronic matrix elements of
bilocal operators containing both straight and staple-shaped gauge
connections; these determine the Ji and the Jaffe-Manohar definitions
of parton OAM, respectively.
\item[Timeline:]
\item[2016] First evaluation of both Ji and Jaffe-Manohar quark OAM from
GTMD observables.
\item[2018] Control of systematic uncertainties in the momentum transfer
dependence of the relevant GTMD through a direct derivative method,
leading to quark angular momentum sum rule being satisfied.
\item[2021] Evaluation of quark spin-orbit correlations in the proton.
\item[2023] Evaluation of quark OAM in the proton via twist-three GPDs.
\item[2024] Enhancing control of systematic uncertainties in Lattice QCD
evaluations of quark OAM; quark mass dependence, excited state effects.
\item[2027] Evaluation of gluon OAM.
\end{itemize}

% Keh-Fei Liu
\subsection{Neutrino-nucleon Scattering from the Hadronic Tensor}
\begin{itemize}
    \item[Motivation.] The formalism of hadronic tensor in lattice QCD~\cite{Liu:1993cv,Liu:1999ak} offers an opportunity to directly evaluate the cross-sections of neutrino-nucleon scattering in the elastic and resonance regions and potentially in the shallow- and deep-inelastic regions as well. The lattice calculation of the neutrino-nucleon scattering cross-sections can be an input for the neutrino-nucleus scattering calculation. The latter is needed for neutrino experiments at DUNE to provide theoretical constraints when analyzing data from neutrino oscillation experiments.
    \item[Long term goal.] Calculate precise neutrino-nucleon scattering cross-sections in all the neutrino energy regions from the threshold to 7 GeV. This should help improve the theoretical understanding of the neutrino-nucleus scattering cross-sections and maximize the discovery potential of neutrino oscillation experiments.
    \item[Method.]  Calculation of matrix elements using four-point correlation functions with vector and axial two-current insertions to the nucleon propagator at different space-time positions to obtain the Euclidean hadronic tensor. Bayesian reconstruction will be employed to solve the inverse problem and obtain the spectral decomposition for the hadronic tensor in the Minkowski space.
    \item[Timeline:]
    \item[1993,1999] The hadronic tensor was formulated in the Euclidean path-integral formalism~\cite{Liu:1993cv}. A new parton degree of freedom -- connected sea partons, was discovered and the operator product expansion was obtained~\cite{Liu:1999ak}.
  \item[2017] The parton evolution equation is generalized to include the connected-sea partons~\cite{Liu:2017lpe}.
    \item[2019] First exploratory calculation of the hadronic tensor was carried out using Backus-Gilbert, Maximum Entropy and Bayesian Reconstruction methods are studied in solving the inverse problem~\cite{Liang:2019frk}.
    \item[2020] Plenary talk at 2019 Lattice Conference which checked the higher-twist contribution and verified that the hadronic tensor matrix element for the elastic scattering is the product of elastic form factors from the three-point correlators~\cite{Liang:2020sqi}.
    \item[2020] It is shown that the parton degrees of freedom from the hadronic tensor are the same as those in the quasi- and pseudo-PDFs~\cite{Liu:2020okp}.
    \item[2021] The connected-sea partons have been incorporated in the global analysis of PDFs for the first time~\cite{Hou:2022ajg}
    \item[2023] Calculation of nucleon elastic and $N\to N^*$ transition form factors and corresponding longitudinal helicity amplitude are completed. A manuscript is being prepared for publication.
    \item[2024] Calculation of the $\Delta$ contribution in the neutrino-nucleon scattering. Explore large momentum transfer
    to study the shallow- and deep-inelastic scattering regions.
    \item[2025-2026] High statistics determinations of the resonance contributions on multiple ensembles and exploration of
    the unpolarized, longitudinal and transversely polarized PDFs.
    \item[2027-] Exploration of generalized parton distribution functions through off-forward matrix elements.
\end{itemize}

% William Jay
\subsection{Hadronic Tensor in the Resonance Region}
\begin{itemize}
    \item[Motivation.] The resonant structure of hadrons in kinematic regions with many open channels is a difficult non-perturbative problem in QCD with relevance to accelerator-based neutrino experiments like DUNE as well as the experimental program at the electron-ion collider.
    Recent theoretical and algorithmic advance in lattice QCD make calculations of the hadronic tensor for fully inclusive resonant scattering a realistic possibility for the first time.
    \item[Long term goal.]
    Benchmarking lattice QCD techniques against the well-measured electromagnetic resonant structure functions of the nucleon.
    QCD predictions for the axial resonant structure functions of the nucleon, to ground our understanding of neutrino-nucleus scattering in QCD.
    \item[Method.] Proof-of-concept calculations are underway using the simpler system of the pion instead of the nucleon.
	The calculation involves four-point correlations functions, which are analytically continued from Euclidean space using novel spectral reconstruction techniques.
\item[Timeline:]
    \item[2017] Modern theoretical understanding of fully inclusive scattering via smeared spectral functions.~\cite{Hansen:2017mnd}
    \item[2019] Algorithm for practical spectral reconstruction.~\cite{Hansen:2019idp}
    \item[2023] Improved algorithm for spectral reconstruction, with rigorously quantified systematic uncertainties.
    \item[2021-2023] Ongoing proof-of-concept calculations for the electromagnetic hadronic tensor of the pion underway.
    \item[2027] First QCD calculations of the hadronic tensor of the nucleon.
\end{itemize}

\section{Nuclear structure}\label{sec:nucstruct}

% Will Detmold
\subsection{Nuclear and hyper-nuclear interactions}

\begin{itemize}
    \item[Motivation.] Nuclear forces at high density are relevant for understanding  the internal structure  of neutron stars and the dynamics of their mergers. Determinations of these forces are relevant to astrophysical observations from NICER and LIGO/Virgo and in experiments aiming to constrain  these interactions such as CREX/PREX at JLab.
    
    \item[Long term goal.] Fundamental understanding of few-nucleon interactions from QCD to provide a predictive framework for nuclear physics. 
    
    \item[Method.] Scattering
      amplitudes computed from finite-volume energy spectrum. Computation of hadronic two-point functions
      using a large set of multi-hadron operators.
      
\item[Timeline:]
    \item[2006] First QCD calculation of $NN$ interactions.
    \item[2012] Calculation of hyperon-nucleon scattering.
    \item[2013] First calculation of nuclear and hypernuclear bindings up to $A=4$.
    \item[2019] Development of automatic code-generation for multi-nucleon correlation functions, enabling much more sophisticated 
    \item[2021] Large variational study of $NN$ interactions.
    \item[2025] Three-nucleon spectrum determined using large operator set.
    \item[2027] Fully-controlled calculations of hyperon-nucleon scattering phase shifts
    
\end{itemize}


% Will Detmold
\subsection{Structure of light nuclei}

\begin{itemize}  
    \item[Motivation.] Much is known about the structure of nuclei, but other aspetcs are mysterious when viewed from the point of view of QCD. The recent interest in short-range correlations in nuclei and their relation to the EMC effect has spawned many experiments at JLab and elsewhere; having a QCD understanding of this physics is crucial. 
    
    \item[Long term goal.] Describe nuclear structure from the underlying Standard Model. 
    
    \item[Method.] Matrix elements corresponding to appropriate operators computed using background field methods.
      
\item[Timeline:]
    \item[2014] First QCD calculation of magnetic moments and polarizabilities of light nuclei.
    \item[2015] QCD calculation of the slow-neutron capture cross section, $np\to d\gamma$.
\item[2018] Calculation of gluon momentum fraction in light nuclei.
    \item[2020] Calculation of the quark momentum fraction in light nuclei including $^3$He. 
    \item[2025] Calculation of EM form factors of nuclei.
\end{itemize}

\subsection{Nuclear matrix elements for intensity frontier experiments}

\begin{itemize}  
    \item[Motivation.] Many intensity-frontier experiments use nuclei as targets in order to increase cross-sections. These experiments include dark-matter direct-detection experiments such as XENON-100, neutrino-nucleus interaction experiments such as DUNE and current and future neutrinoless double-$\beta$ decay search experiments including nEXO.
    
    \item[Long term goal.]  Provide Standard Model nuclear matrix elements for interpretation of intensity frontier experiments.
    
    \item[Method.] 
    Calculation of nuclear matrix elements using three-point correlation functions.
    
\item[Timeline:]
\item[2016] Calculation of neutrinoful double-$\beta$ decay matrix elements revealing importance of isotensor axial polarizability.
    \item[2018] Calculation of scalar matrix elements in light nuclei relevant for dark matter interactions 
    \item[2021] Calculation of the axial charge of the triton, providing two-body EFT input for neutrino-nucleus scattering.
    \item[2024] Calculation of short- and long-distance neutrinoless double-$\beta$ decay matrix elements for $nn\to pp$.
\end{itemize}


\section{Extreme matter}\label{sec:extreme}

% Peter Petreczky
\subsection{QCD phase diagram}\label{sec:qcdphase}
\begin{itemize}
  \item[Motivation.] The aim of ultra-relativistic heavy ion
    collisions is to study strongly interacting matter at high
    temperatures. There is a large ongoing experimental program at
    RHIC in BNL and LHC at CERN dedicated to this. Lattice QCD
    calculations provide the necessary input for theoretical
    interpretation of heavy ion experiments. 

 \item[Long term goal.] The goal of lattice QCD calculations is to
      map out the QCD phase diagram as function of the temperature at
      zero and moderately high baryon density, calculate the QCD
      equation of state and estimate spectral functions that are
      needed for heavy flavor and electromagnetic probes of the matter
      produced in ultra-relativistic heavy ion
      collisions. \item[Method.] Calculations are performed using
      highly improved staggered quark (HISQ) action in 2+1 flavor QCD
      with physical strange quark mass, $m_s$ and various light quark
      masses, $m_l$,  around its physical value of $m_l=m_s/27$. In
      some case calculations are performed for larger value of
      $m_l$. For the calculations of the spectral function sometimes
      values $m_l=m_s/5$ are used while for exploring the chiral
      aspects of the transition smaller than physical values of $m_l$
      are used. 

  \item[Timeline:]
   \item[2011-2014] The HotQCD collaboration successfully determined
     the chiral crossover temperature and QCD equation of state in the
     continuum limit. 
   \item[2014-2021] The HotQCD collaboration determined the chiral
     crossover temperature, the QCD equation of state and fluctuations
     of conserved charges at non-zero density, which was essential
     input for the RHIC BES program. In particular these results have
     been used by the BEST DOE topical collaboration.
   \item[2019-2023] Calculations of the bottomonium properties and the
     heavy quark potential have been performed at nearly physical
     light quark masses on lattices with temporal extent
     $N_{\tau}=12$. The heavy quark diffusion coefficient has been
     calculated in 2+1 flavor QCD for the first time at
     $m_l=m_s/5$. All these calculations are important for the
     interpretation of the open and hidden heavy flavor production in
     heavy ion experiments at RHIC and LHC, including START, sPHENIX,
     ALICE, CMS and ATLAS.
   \item[2023-2027] Our goal in the future is to calculate quarkonium
     in-medium properties and the heavy quark diffusion coefficient in
     the continuum limit for physical light quark masses.
   \item[2023-2035] Heavy flavor probes will be one of the driving
     themes of heavy ion experiments at RHIC and LHC. The sPHENX and
     STAR experiments will collect new data in 2023-2025. The analysis
     of these data will probably last till 2028. The LHC heavy ion
     experiments are planned to run till at least 2035. 
\end{itemize}


\section{Energy Frontier}\label{sec:energy}


INTENSITY FRONTIER  ( QCD strong coupling, as defined at the Z-pole)

\begin{itemize}
	\item[Motivation.]  The most precise lattice determination of the QCD strong coupling $\alpha_s$  comes from the european Alpha collaboration but  it should require independent and important  cross-validation. Two  USQCD  groups are working toward  this goal.
	\item[Long term goal.] The QCD strong coupling $\alpha_s$, as defined at the Z-pole, is equivalent to the long term goal to determine $\Lambda_{\overline{MS}}$  in physical units with very high precision. Credible and high precision cross-validation  of this goal  is motivated by the emergence of  the required new lattice  technology for the long-term goal.
	\item[Method.]  The application of gradient flow based scale-dependent renormalization of $\alpha_s(\mu=1/\sqrt{8t})$ at flow time scale $t$ on lattices extended to  infinite space-time provides the new technology. 
	\item[Timeline:]
	\item[2012] The LatHC collaboration introduces the gradient flow based step $\beta$-function where the scale is set by the size of the physical volume Fodor:2012td. The method  has been extended  applied to high precision QCD and to BSM models. 
	\item[2017-2018] The LatHC collaboration introduces and tests in the sextet BSM model the  gradient flow based  $\beta$-function for the renormalized coupling at flow time scale $t$ where the lattice is extended to infinite volume.  The method is tested at a single renormalized gauge coupling \cite{Fodor:2017die}. 
	\item[2019-2020] The new paradigm was extended to predict the entire $\beta$ function from UV to IR. This approach was tested in two-flavor QCD with massless fermions\cite{Hasenfratz:2019hpg} and in multiflavor QCD with ten and twelve flavors.
	\item[2022-2023] In the SU(3) Yang-Mills limit of quenched QCD it was shown by two USQCD groups that the new method is a competitive high-precision match to the earlier method of the Alpha collaboration \cite{Hasenfratz:2023bok,Wong:2023jvr}. The combined high accuracy is in signifcant tension with any other lattice method.
		\item[2023-2024]  Ongoing and future work of two USQCD groups is trying the reproduce the success of their quenched QCD results with three massless quark flavors in QCD to achieve the stated  goal. 
	
\end{itemize}

\newpage

ENERGY FRONTIER  ( sextet model and ten-flavor QCD)

\begin{itemize}
	\item[Motivation.]  Gauge theories with massless fermions and large number of flavors, or in higher fermion representations, become near-conformal with the signal of an emergent dilaton. The effetive field theory description of this dilaton is an important challenge to understand as one of the most important clues to the conformal window.
   	\item[Method.] The effective field theory description of this dilaton is an important challenge to understand as one of the most significant clues to the conformal window. The  effective dilaton theories of the two models are tested in the p-regime and epsilon regime of the chirally broken phase near conformality.
	\item[Timeline:]
\item[2018-2020]   	The LatHC collaboration presents dilaton effecive field theory analysis  of  the sextet model. Similar tests were obtained in the eight-flavor model based on the data of the LSD collaboration.
\item[2023-2024]   There are ongoing tests in the sextet model and in the ten-flavor model to further develop the effective dilaton theory based on the anomalous dimension of composite four-fermion operators.

   	
\end{itemize}
\subsection{Running coupling, $\alpha_s(M_Z)$, and non-perturbative renormalization of QCD}
\begin{itemize}
   \item[Motivation.] Determining the QCD running coupling, the $\Lambda$ parameter, and the strong coupling constant $\alpha_s$ are essential to connect lattice calculations to continuum renormalization schemes. 
   \item[Long term goal.] Lattice calculations can determine the $\Lambda_{QCD}$  from the infrared lattice scale to high energies where perturbative connection to the ${\overline {MS}}$ scheme is possible. The goal is to determine the renormalization group $\beta$ function for the $N_f=0$ Yang-Mills system and for  $N_f=2$ and 3 flavor QCD. Using the decoupling method these determinations can be connected, providing further verification of the flavor number dependence.
   \item[Method.] In recent years the continuous $\beta$ function method, originally developed to study the RG properties of near-conformal systems, proved to be a very reliable approach. Several groups use the method to achive the stated goals. Calculations with  different lattice actions and lattice fermions  provide an other level of control over this important quantity.
\item[Timeline:] 
   \item[2015-2021] Development of the continuous $\beta$ function method (see next section)
   \item[2021-2023] First results for the $N_f=0$ system are published. Calculations of the $N_f=2$ and 3 systems have started. 
   \item[2023-2025] First published result for $N_f=2$ is expected in 2023, followed by more precise calculations for both $N_f=2$ and 3.
   
\end{itemize}

\subsection{Renormalization group properties beyond - QCD systems }

\begin{itemize}
    \item[Motivation.]
     Understanding the nature of SU(3) gauge theories and how infra-red
     fixed points in many flavor systems arise are important for composite Higgs models. The anomalous dimension of composite fermions is needed to explain fermion mass generation of Standard Model fermions.  
    \item[Long term goal.] Studying SU(3) gauge fermion-systems and explore novel gradient flow methods
    \item[Method.] Generation dynamical gauge field configurations with
     SU(3) gauge group and $N_f$ fundamental flavors. Calculation of
     the step-scaling function, the discrete analog of the
     renormalization group (RG) $\beta$ function. In addition we
     explore the connection of gradient flow to the RG flow and propose
     a new method to calculate the continuous $\beta$ function,
     extract anomalous dimensions, and obtain renormalization Z factors
    \item[Timeline:]
    \item[2017-2019] Calculation of the step-scaling $\beta$ function
     for SU(3) with $N_f=12$ fundamental flavors
    \item[2018-2020] Calculation of the step-scaling $\beta$ function
     for SU(3) with $N_f=10$ fundamental flavors
    \item[2020] Development of the continuous $\beta$ function
    \item[2020-2022] Calculation of the step-scaling $\beta$ function
     for SU(3) with $N_f=8$ fundamental flavors
    \item[2021] Proposal of a new gradient flow based renormalization
     scheme
    \item[2021-2022] Calculation of the step-scaling $\beta$ function
     for SU(3) with $N_f=6$ and 4 fundamental flavors
    \item[2020-2023] New concept to use the gradient flow to determine
     the $\Lambda$ parameter for pure gauge systems
    \item[2021-2025] Establishing the viability of GF renormalization
\end{itemize}

\subsection{Composite Higgs models}
\begin{itemize}
    \item[Motivation.] Exploring composite Higgs models and make
     predictions for resonances to be eventually discovered by ATLAS
     and CMS
    \item[Long term goal.] Understanding the viability wether the
     observed 125 GeV Higgs boson can be understood as an excitation of
     a new strongly interacting sector
    \item[Method.] Large scale numerical simulations of (near)-
     conformal gauge theories or theories close to an IRFP
    \item[Timeline:]
    \item[2014-2016] Proposing and exploring mass-split models using a
     setup with four light and eight heavy flavors
    \item[2017-???] Studying improved mass-split system with 4 light
     and six heavy flavors
\end{itemize}

\subsection{Dark Matter}

\subsection{Theoretical developments}

\newpage

\appendix

%\bibliographystyle{unsrt}
\bibliographystyle{apsrev}
\bibliography{ref.bib}

\end{document}
