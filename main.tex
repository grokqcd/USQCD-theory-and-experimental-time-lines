
  % % % % & pdflatex


%\documentclass[aps,hyperpdf]{revtex4-2}
%\documentclass[aps]{revtex4-2}


%\documentclass[prd,showpacs,showkeys,preprintnumbers,floatfix, 
%nofootinbib%,superscriptaddress
%]{revtex4-2}

%\documentclass[12pt]{article}
\documentclass[12pt,hyperpdf]{article}

%\pdfoutput=1 

\usepackage[utf8]{inputenc}
\usepackage{graphicx}
\usepackage{amsmath}
\usepackage{amsfonts} 

\usepackage[nottoc]{tocbibind}

\usepackage[numbers]{natbib}

\usepackage{hyperref}

%\hypersetup{
%    colorlinks   = true,
%    citecolor    = blue
%}

\hypersetup{ 
    pdfnewwindow=true,         % links in new window
    colorlinks=true,           % false: boxed links; true: colored links
    linkcolor=blue,            % color of internal links
    citecolor=blue,            % color of links to bibliography
    filecolor=blue,            % color of file links
    urlcolor=blue,              % color of external links
    hypertexnames=blue
} 



\setlength{\textwidth}{480pt}
\setlength{\textheight}{650pt}
\setlength{\oddsidemargin}{0pt}
\setlength{\topmargin}{-60pt}

\title{USQCD theory and experimental time lines}
\author{USQCD Collaboration}
\date{May 2024}


\begin{document}


% \begin{abstract}
%We present a timeline of the USQCD theory computational program along
%with experimental timelines.
%\end{abstract}


\maketitle
%\thispagestyle{empty}

\begin{center}
{\bf URL:}\ \ \url{https://github.com/grokqcd/USQCD-theory-and-experimental-time-lines}
\end{center}



\setcounter{page}{1}
%\pagenumbering{roman}

\tableofcontents

\clearpage
\setcounter{page}{1}

\section{Intensity Frontier}\label{sec:intensity}

% Tom Blum
\subsection{Muon $g-2$}
\begin{itemize}
    \item{\bf Motivation:} The muon anomalous magnetic moment, or
      $g-2$, will be measured by the E989 experiment at Fermilab to
      the one-per=mille precision level and thus offers an exceptional
      opportunity to test the standard model if theory can match this
      precision. 
    \item{\bf Long term goal:} The Muon $g-2$ Theory Initiative aims
      to provide a consensus theory value at the same level of
      precision as the E989 experiment. Only QCD, or hadronic,
      contributions need improvement to reach this goal.      
    \item{\bf Method:} Lattice  calculations of correlation functions
      in QCD for the hadronic vacuum polarization (HVP) (two-point)
      and hadronic light-by-light scattering (HLbL) (four-point) are
      combined with perturbative QED at $O(\alpha^2)$ and
      $O(\alpha^3)$, respectively, to obtain the leading hadronic
      contributions.  
\item{\bf Timeline:} \hfill [last updated May 2023]
\begin{itemize}
    \item{\bf 2021} Muon $g-2$ Theory Initiative whitepaper~\cite{Aoyama:2020ynm} released with consensus Standard Model Value based on data-driven HVP and data-driven+lattice HLbL contributions.
    \item{\bf 2021} E989 announces first results with 0.48 ppm precision. Combined with BNL 821 (0.54 ppm), yields 4.2 standard deviation discrepancy with theory.
    \item{\bf 2021} BMW collaboration computes HVP contribution to 0.75\%, consistent with experiment. 
    \item{\bf 2022} USQCD groups Fermilab Lattice/HPQCD/MILC and RBC/UKQCD, $\chi$QCD, Aubin {\it et al.}, and several groups from Europe obtain agreement on the HVP intermediate window which is four standard deviations discrepant with the data-driven value.
    \item{\bf 2023} Several groups to update total HVP contribution, including isospin corrections, with sub-percent errors.
    \item{\bf 2023} RBC updated HLbL contribution, reaching 12\% error.
    \item{\bf 2023} E989 to announce run 2 and 3 combined result, improving precision by a factor of two.
    \item{\bf 2024} RBC to reduce error on HLbL contribution by a factor of two.
    \item{\bf 2024-25} RBC/UKQCD and FHM reach 1-2 per-mille goal on HVP contribution. 
    \item{\bf 2025-26} E989 releases final result with expected precision of $\sim 1.4$ ppm
\end{itemize}
\end{itemize}

% Carleton Detar
\subsection{$B$-meson anomalies and CKM unitarity tests}
\begin{itemize}
    \item{\bf Motivation:} $B$-flavor physics offers excellent opportunities for stringent tests of the Standard Model, tests that may lead to the discovery of new physics. In conjunction with refined experimental results, precision lattice-QCD calculations yield precise values of the CKM matrix elements, allowing checks of CKM unitarity and tests of lepton universality. Measurements of rare semileptonic $B$-meson decays (such as $B\to\pi\ell^+\ell^-$, $B\to K\ell^+\ell^-$) are promising channels for new-physics searches because their rates are suppressed in the Standard Model.
    \item{\bf Long term goal:} Lattice-QCD calculations go hand-in-hand with experiment.  Thus the long-term goal is to keep pace with the improving experimental precision of each quantity under study.
    \item{\bf Method:} Most calculations work with a combination of hadronic two- and three-point functions, measured at several lattice spacings and quark masses, and with a variety of current operators. 
\item{\bf Timeline:} \hfill [last updated May 2024]
\begin{itemize}
    \item{\bf 2014} Calculation of leptonic $B$, $B_s$, $D$, and $D_s$ decay constants to sub-percent accuracy.
    \item{\bf 2015} First measurement by CMS/LHCb of the decay $B_s \to \mu^+\mu^-$.
    \item{\bf 2021} Calculation of $R(D*)$ to 4\%.
    \item{\bf 2022} Precise calculation of $|V_{cs}|$ with a QCD error comparable to the current experimental precision and of $|V_{cd}|$ with a QCD error twice that of experiment~\cite{FermilabLattice:2022gku}.
    \item{\bf 2024} Calculation of $|V_{cb}|$ via $B\to D\ell\nu$ and $|V_{ub}|$ via $B\to \pi\ell\nu$.
    \item{\bf 2024} Calculation of $B\to\pi \ell^+\ell^-$ and $B\to K \ell^+\ell^-$ to 2\%.
    \item{\bf 2026} LHCb End of Run 3 data taking. Beginning of Long shutdown 3.
    \item{\bf 2027} Calculation of $|V_{cb}|$ to the sub-percent level via $B\to D^*\ell\nu$.
    \item{\bf 2027} Calculation of $R(D*)$ to 1\%.
    \item{\bf 2027-28} Belle II Long shutdown 2.
%    \item{\bf ??} Belle II
%    \item{\bf ??} LHCb
%    \item{\bf ??} BES III
\end{itemize}
\end{itemize}

% Stefan Meinel
\subsection{Flavor physics with heavy baryons}
\begin{itemize}
   \item{\bf Motivation:} Semileptonic decays of charm and bottom baryons provide determinations of CKM matrix elements, tests of lepton-flavor universality, and constraints on flavor-changing neutral-current Wilson coefficients. The nonzero spins of the initial and final-state baryons make the decay amplitudes sensitive to all possible operator structures appearing in the weak effective Hamiltonian and provide a large number of angular observables that can be used to disentangle these structures.
   \item{\bf Long term goal:} Using lattice QCD, calculate the form factors for all heavy-baryon semileptonic decays whose decay rates are being measured at the Large Hadron Collider and other experiments. Improve the calculations over time to keep up with the increasing experimental precision.
   \item{\bf Method:} The relevant baryon two-point and three-point functions are computed for a wide range of source-sink separations to obtain good control over excited-state contamination. Light and strange forward propagators with smeared sources can be reused for multiple processes. The three-point functions use heavy-quark sequential propagators.
\item{\bf Timeline:} \hfill [last updated May 2024]
\begin{itemize}
   \item{\bf 2015:} Calculation of the $\Lambda_b\to p\ell^-\bar{\nu}$ and $\Lambda_b\to \Lambda_c\ell^-\bar{\nu}$ form factors and SM predictions for decay rates and $R(\Lambda_c)$ \cite{Detmold:2015aaa}
   \item{\bf 2015} Determination of $|V_{ub}/V_{cb}|$ from $\Lambda_b\to p\mu^-\bar{\nu}$ and $\Lambda_b\to \Lambda_c\mu^-\bar{\nu}$ decays by LHCb \cite{Aaij:2015bfa} using the form factors from Ref.~\cite{Detmold:2015aaa}
   \item{\bf 2015} Measurement of the $\Lambda_b \to \Lambda \mu^+\mu^-$ differential branching fraction and angular observables by LHCb \cite{Aaij:2015xza}
   \item{\bf 2015} Measurement of the $\Lambda_c \to \Lambda e^+\nu$ branching fraction by BESIII \cite{BESIII:2015ysy}
   \item{\bf 2016} Calculation of the $\Lambda_b \to \Lambda \ell^+\ell^-$ form factors and SM predictions \cite{Detmold:2016pkz}
   \item{\bf 2016} Measurement of the $\Lambda_c \to \Lambda \mu^+\nu$ branching fraction by BESIII \cite{Ablikim:2016vqd}
   \item{\bf 2016} Calculation of the $\Lambda_c \to \Lambda \ell^+\nu$ form factors and determination of $|V_{cs}|$ \cite{Meinel:2016dqj}
   \item{\bf 2017} Measurement of the shape of the $\Lambda_b\to\Lambda_c \mu^- \bar{\nu}$ differential decay rate by LHCb \cite{Aaij:2017svr} in agreement with the LQCD prediction from Ref.~\cite{Detmold:2015aaa}
   \item{\bf 2017} Calculation of the $\Lambda_c \to n \ell^+\nu$ and $\Lambda_c \to p \ell^+\ell^-$ form factors and SM predictions \cite{Meinel:2017ggx}
   \item{\bf 2017} Search for $\Lambda_c \to p \mu^+\mu^-$ by LHCb \cite{Aaij:2017nsd}
   \item{\bf 2018} Measurement of $\Lambda_b \to \Lambda \mu^+\mu^-$ angular moments by LHCb \cite{Aaij:2018gwm}
   \item{\bf 2020} Calculation of the $\Lambda_b \to \Lambda^*(1520) \ell^+\ell^-$ form factors and SM predictions \cite{Meinel:2020owd}
   \item{\bf 2021} Calculation of the $\Lambda_b \to \Lambda_c^*(2595,2625)\ell^-\bar{\nu}$ form factors and SM predictions \cite{Meinel:2021rbm}
   \item{\bf 2021} Calculation of the $\Lambda_c \to \Lambda^*(1520) \ell^+\nu$ form factors and SM predictions \cite{Meinel:2021mdj,Meinel:2021grq}
   \item{\bf 2022} Measurement of $R(\Lambda_c)$ by LHCb \cite{LHCb:2022piu}
   \item{\bf 2022} Measurement of the $\Lambda_c \to \Lambda^*(1520) e^+\nu$ branching fraction by BESIII \cite{BESIII:2022qaf}
   \item{\bf 2022} Measurement of differential branching fraction and angular analysis of $\Lambda_c \to \Lambda e^+\nu$ by BESIII \cite{BESIII:2022ysa}
   \item{\bf 2023} Measurement of the $\Lambda_b \to \Lambda^*(1520) \mu^+\mu^-$ differential branching fraction by LHCb \cite{LHCb:2023ptw}
   \item{\bf 2023} Measurement of differential branching fraction and angular analysis of $\Lambda_c \to \Lambda \mu^+\nu$ by BESIII \cite{BESIII:2023jxv}
   \item{\bf 2023-2025} LHCb Run 3
   \item{\bf 2024} Calculation of $\Xi_c\to\Xi\ell^+\nu$ form factors and SM predictions
   \item{\bf 2025} Calculation of $\Xi_b\to\Xi\ell^+\ell^-$ form factors and SM predictions
   \item{\bf 2025} Next-generation calculation of $\Lambda_b\to p\ell^-\bar{\nu}$, $\Lambda_b\to \Lambda_c\ell^-\bar{\nu}$, and  $\Lambda_b \to \Lambda \ell^+\ell^-$ form factors and SM predictions
   \item{\bf 2025-2030} BESIII additional runs at the $\Lambda_c\overline{\Lambda_c}$ threshold
   \item{\bf 2026-2028} LHCb Upgrade 1b
   \item{\bf 2029-2032} LHCb Run 4
\end{itemize}
\end{itemize}

% Norman Christ
\subsection{$K_L$-$K_S$ mass difference}
\begin{itemize}
    \item{\bf Motivation:} This is a precisely measured quantity that is highly sensitivity to new physics 
    at an energy scale of $10^4$ TeV.
    \item{\bf Long term goal:} Calculate $K_L - K_S$ mass difference with
    a precision exceeding the current experimental value of $3.484
    \pm 0.007 \times  10^{-12}$ MeV.
    \item{\bf Method:} Well understood lattice QCD technique with no
    recognized limitation to the control of all systematic
    errors. GIM cancellation is essential and treatment of the
    charmed quark using QCD perturbation theory introduces 36\%
    errors. The lattice calculation is made difficult by the
    requirement of a lattice spacing smaller than the charm quark
    Compton wave length and a volume sufficiently large to accurate
    treat physical mass pions. 
\item{\bf Timeline:} \hfill [last updated May 2024]
\begin{itemize}
    \item{\bf 2010} Accurate KTeV result~\cite{KTeV:2010sng}.
    \item{\bf 2010-2014} Lattice QCD method devised and first results 
    obtained.~\cite{Christ:2012se, Bai:2014cva}.
    \item{\bf 2014-2021} First results with physical masses and $1/a = 2.38$ MeV 
    obtained, presence of large discretization errors recognized~\cite{Bai:2018lrm, 
    Wang:2021twm}.
    \item{\bf 2021-2023} Discretization errors studied and evidence of $a^2$ scaling 
    found.
    \item{\bf 2023-2024} Calculation of the long-distance contribution to
    $\varepsilon_K$ with $1/a = 2.38$ MeV will include extension of
    earlier $\Delta M_K$, calculation, increasing statistics.
    \item{\bf 2024-2026} Calculation of $\Delta M_K$ with $1/a = 2.8$ GeV
    giving continuum limit result with $20$\% errors. 
    \item{\bf 2027-2030} Move to $2+1+1$ flavors and $1/a = 3.0$, $4.0$
    GeV and possibly 5 GeV with $10$\% errors. 
\end{itemize}
\end{itemize}

% Norman Christ
\subsection{Long distance contribution to $\epsilon_K$}
\begin{itemize}
    \item{\bf Motivation:} The indirect CP violation parameter $\epsilon_K$ is precisely 
    measured and sensitive to BSM sources of CP violation at high energies. Its 
    accurate standard model (SM) prediction provides a critical test of the KM 
    theory of CP violation.  While 95\% of the SM prediction comes from short 
    distance and is computed to an accuracy approaching 1\%, there is a 5\% 
    contribution from long distances on the scale of the charm quark Compton 
    wavelength and larger that requires a lattice QCD calculation.
    \item{\bf Long term goal:} Calculate the long-distance contributions to $\epsilon_K$ 
    to 10\% precision, sufficient to match or exceed the current experimental 
    $\epsilon_K$ precision: $|\epsilon_K|=2.228\pm 0.011 \times 10^{-3}$~\cite{ParticleDataGroup:2022pth}.
    \item{\bf Method:} Well understood lattice QCD technique with no recognized 
    limitation to the control of all systematic errors. GIM cancellation is essential 
    but a logarithmic divergence in the effective long-distance theory requires the 
    perturbative calculation of a low energy constant which becomes systematically 
    more accurate as the matching scale is increased above the charm quark mass.
\item{\bf Timeline:} \hfill [last updated May 2023]
\begin{itemize}
    \item{\bf 2000} KTeV and NA48 measurements of $\epsilon'$.
    \item{\bf 2014-2017} Lattice QCD method devised and first results 
    obtained~\cite{Christ:2015phf}. 
    \item{\bf 2023-2024} First calculation with physical quark masses but relatively 
    coarse $1/a=2.38$ GeV lattice spacing (combined with an extension of the
    $\Delta M_K$ calculation with increased statistics).
    \item{\bf 2024-2026} Calculation including $\Delta M_K$ with $1/a = 2.8$ GeV 
    giving continuum limit result with $20$\% errors. 
    \item{\bf 2027-2030} Move to $2+1+1$ flavors and $1/a = 3.0$, $4.0$
    GeV and possibly 5 GeV with $10$\% errors. 
\end{itemize}
\end{itemize}


% Tom Blum/Norman Christ
\subsection{Direct $CP$-violation in kaon decays ($\varepsilon'$)}
\begin{itemize}
    \item{\bf Motivation:} Highly sensitive to new sources of CP that may explain the matter/antimatter asymmetry in the observable Universe. Experimental result with ${\cal O}(10\%)$ precision already available.
    \item{\bf Long term goal:} Calculate $\varepsilon'$ to a precision exceeding the current 10\% experimental value.        
    \item{\bf Method:} Lattice calculation of $K\to\pi\pi$ matrix elements using 3-flavor weak effective theory in finite-volume allows extraction of infinite-volume amplitudes, $A_2$ \& $A_0$, through Lellouch-L\"uscher methods.  Use two complementary finite-volume treatments: i) Antiperiodic ($A_2$) and novel ``G-parity'' ($A_0$) boundary conditions (BCs). Requires bespoke G-parity ensembles.  Energy of two pion ground state equals $M_K$. ii) Periodic boundary conditions.  Multiple finite volume energies measured using Generalized Eigenvalue Problem (GEVP) method~\cite{Luscher:1990ck,Bulava:2011yz}.  Can use existing ensembles and can add E\&M.  Together both methods allow study of finite-volume errors.  Non-perturbative renormalization at high energies minimizes systematic error when matching to perturbation theory describing weak interaction physics. Vacuum contributions to $A_0$ require large statistics to adequately resolve. Future calculations aim to reduce systematic errors by incorporating electromagnetism and isospin-breaking effects, and an active charm quark.  Well-understood lattice QCD technique with no recognized limitation to the control of all systematic errors. 
\item{\bf Timeline:} \hfill [last updated March 2024]
\begin{itemize}
    \item{\bf 1999} First definitive observations of kaon direct CP-violation at KTeV (FermiLab) and NA48 (CERN).
    \item{\bf 2001} Final NA48 experimental result published~\cite{NA48:2001bct}.
    \item{\bf 1985-2002} Early lattice calculations using quenched QCD and chiral perturbation theory (ChPT) obtained results with large, uncontrolled systematic errors.
    \item{\bf 2004} Lattice calculation with dynamical QCD and ChPT~\cite{Li:2008kc}. Large systematic errors due to ChPT discouraged continued use of this approach.
    \item{\bf 2011} Final KTeV experimental result published~\cite{KTeV:2010sng}. Combining experimental results gives current, best determination ${\rm Re}(\varepsilon'/\varepsilon)=16.6(2.3)\times 10^{-4}$.
    \item{\bf 2011} Development of lattice approach for computing $K\to\pi\pi$ decays directly~\cite{Blum:2011pu}.
    \item{\bf 2012} First {\it ab initio}, physical calculation of $A_2$~\cite{Blum:2011ng,Blum:2012uk}.
    \item{\bf 2015} First continuum-limit calculation of $A_2$~\cite{Blum:2015ywa}. First {\it ab initio}, physical calculation of $A_0$ and $\varepsilon'$ using G-parity BCs~\cite{RBC:2015gro}.
    \item{\bf 2020} Improved calculation of $A_0$ and $\varepsilon'$ with G-parity BCs obtained substantially smaller statistical errors 
     and better control over systematic errors~\cite{RBC:2020kdj}. This, latest result agrees with experiment but has ${\sim}4\times$ 
     the total error.
    \item{\bf 2020-2025} Repeat G-parity BC calculation of $A_0$ on finer lattices to reduce/remove ${\cal O}(12\%)$ discretization 
    systematic error.
    \item{\bf 2023} First calcultion with periodic boundary conditions (PBC).  Results obtained for physical-quark mass 
    $\pi\pi$ scattering~\cite{RBC:2023xqv} and $\varepsilon'$~\cite{RBC:2023ynh} with $a^{-1}=1.02$~GeV.  Results agree with 
    experiment and the G-parity result but with less precision because of fewer configurations and coarser lattice used.
    \item{\bf 2022-2024} Continue PBC calculations with more configurations and a finer lattice with $a^{-1}=1.38$~GeV to obtain 
    a more precise result with reduced statistical and discretization errors.
    \item{\bf 2020-2026} Develop new methods to incorporate electmagnetism and isospin-breaking effects, reducing dominant 
    ${\cal O}(23\%)$ systematic error.  See E\&M correction project below.
    \item{\bf 2024-2026} Develop approach to incorporating active charm quarks with controlled discretization errors, reducing 
    an ${\cal O}(12\%)$ systematic error.
    \item{\bf 2026-2031} Improved measurements employing new approaches, aiming to match/exceed experimental precision 
    by end of Snowmass '21 period. Potential discovery of tension between experiment and Standard Model may lead to new 
    generation of kaon experiments.
\end{itemize}
\end{itemize}


% Norman Christ/Tom Blum
\subsection{Direct $CP$ violation in $K\to\pi\pi$ decays using periodic boundary conditions}
\begin{itemize}
    \item{\bf Motivation:} Precise SM prediction of $\varepsilon'$, the measure of direct $CP$ violation in $K\to\pi\pi$ decay, which is very sensitive to new physics. 
    \item{\bf Long term goal:} Calculate Re($\varepsilon'/\varepsilon$) with
      a precision exceeding the current experimental value of $1.66(23)\times10^{-3}$, which roughly corresponds to 10\% precision of $A_0$, the decay amplitude with isospin-0 final state.
    \item{\bf Method:} Well understood lattice QCD technique with no
      recognized limitation to the control of all systematic
      errors. The $K\to\pi\pi$ matrix elements with the physical kinematics are to be
      extracted by using the well developed Generalized eigenvalue problem (GEVP)
      method~\cite{Luscher:1990ck,Bulava:2011yz}.  The Lellouch-L\"uscher
      formalism~\cite{Lellouch:2000pv} has been used to relate finite- and
      infinite-volume two-pion states in isospin-symmetric calculations but
      needs to be extended for introduction of electromagnetism, which is
      expected to give a significant ($O(20\%)$) impact on $\varepsilon'$
      but has been absent from earlier lattice calculations.  Effects of the
      charm quark and naive discretization effects, which are also significantly
      uncertain for achieving our precision goal at this point, are to be both
      reduced by calculating on finer lattices.
\item{\bf Timeline:}
\begin{itemize}
    \item{\bf 1999} First definitive observations of kaon direct CP-violation at KTeV (FermiLab) and NA48 (CERN).
    \item{\bf 2001} Final NA48 experimental result published~\cite{NA48:2001bct}.
    \item{\bf 1985-2002} Early lattice calculations using quenched QCD and chiral perturbation theory (ChPT) obtained results with large, uncontrolled systematic errors.
    \item{\bf 2004} Lattice calculation with dynamical QCD and ChPT~\cite{Li:2008kc}. Large systematic errors due to ChPT discouraged continued usage of this approach.
    \item{\bf 2011} Final KTeV experimental result published~\cite{KTeV:2010sng}. Combining experimental results gives current, best determination ${\rm Re}(\varepsilon'/\varepsilon)=16.6(2.3)\times 10^{-4}$.
    \item{\bf 2011} Development of lattice approach for computing $K\to\pi\pi$ decays directly~\cite{Blum:2011pu}.
    \item{\bf 2012} First {\it ab initio}, physical calculation of $A_2$~\cite{Blum:2011ng,Blum:2012uk}.
    \item{\bf 2015} First continuum-limit calculation of $A_2$~\cite{Blum:2015ywa}. First {\it ab initio}, physical calculation of $A_0$ and $\varepsilon'$ using G-parity BCs~\cite{RBC:2015gro}.
    \item{\bf 2020} Improved calculation of $A_0$ and $\varepsilon'$ with G-parity BCs obtained substantially smaller statistical errors and better control over systematic errors~\cite{RBC:2020kdj}. This, latest result agrees with experiment but has ${\sim}4\times$ the total error.
    \item{\bf 2023} Excited two-pion state that corresponds to the final state of energy-conserving $K\to\pi\pi$ decay with periodic boundary conditions (PBC) successfully extracted~\cite{RBC:2023xqv}.  First PBC result for $\varepsilon'$ with $a^{-1}=1.02$~GeV was reported~\cite{RBC:2023ynh}.  It agrees with experiment and the G-parity result, but the precision is worse due to fewer configurations and coarser lattice used.
    \item{\bf 2020-2024} Develop a method to nonperturbatively incorporate the effect of a charm loop of the four-quark operators on the $\Delta I=1/2$ $K\to\pi\pi$ amplitude.
    \item{\bf 2022-2024} Continue PBC calculations with more configurations and on a finer lattice of $a^{-1}=1.38$~GeV to obtain a more precise result with reduced statistical and discretization errors.
    \item{\bf 2020-2026} Develop new methods to incorporate electromagnetism and isospin-breaking effects, reducing dominant ${\cal O}(20\%)$ systematic error.  (See item below.)
    \item{\bf 2024-2027} Repeat calculations of $A_0$ and $\varepsilon'$ on even finer lattices $a^{-1}=1.7, 2.3, 2.7$~GeV to conclude our calculation at the iso-symmetric point. 
    \item{\bf 2026-2030} Improved measurements including isospin-breaking and electromagnetic corrections, aiming to match/exceed experimental precision. Potential discovery of a tension between experiment and the SM may prompt a new generation of $\varepsilon'$ experiments. 
\end{itemize}
\end{itemize}


% Norman Christ
\subsection{Contribution of E\&M and strong isospin breaking to $\varepsilon'$}
\begin{itemize}
    \item{\bf Motivation:} Because of the $\Delta I=1/2$ rule, the isospin breaking corrections to the direct CP violating parameter $\varepsilon'$ are enhanced by a factor of 20, increasing their usual 1\% scale to potentially 20\% and making the uncertainty in these corrections one of the dominant errors in the current standard model prediction of $\varepsilon'$.
    \item{\bf Long term goal:} Calculate the isospin breaking corrections to $\varepsilon'$ to 10\% accuracy removing this as a source of error in the prediction of $\varepsilon'$.
    \item{\bf Method:} The addition of electromagnetism adds substantial barriers to the already challenging lattice calculation of $\varepsilon'$:  i) The $I=0$ and $I=2$ $\pi\pi$ final states mix and the three-particle $\pi\pi\gamma$ state enters making this a complicated, multi-channel decay. ii)  Standard lattice formulations of QED have substantial finite-volume errors while the successful lattice methods to compute $\varepsilon'$ require that the calculation be performed in a volume of limited size.  iii) Adding photons substantially increases the complexity of the quantities being computed, likely requiring the use of computer-generated code.
\item{\bf Timeline:}  \hfill [last updated March 2024]
    \item{\bf 2000} KTeV and NA48 measurements of $\varepsilon'$
    \item{\bf 2017} Non-relativistic method to compute the Coulomb contribution to $\varepsilon'$ devised~\cite{Christ:2017pze}.
    \item{\bf 2019-2021} Non-relativisitic method is generalized to one that is fully relativistic~\cite{Christ:2021guf}.
    \item{\bf 2023-2024} Extend the Coulomb-gauge approach to include transverse radiation.  Estimate the contribution of low-energy $\pi\pi\gamma$ states to $\varepsilon'$.  If this estimate lies below 20\% of expected size of E\&M and isosping breaking effects begin a lattice calculation.
    \item{\bf 2024-2027} Calculation of isospin breaking contributions to $\varepsilon'$ to 30\% accuracy.
    \item{\bf 2027-2030} Calculation of isospin breaking contributions to $\varepsilon'$ to 10\% accuracy.
\end{itemize}
% Norman Christ




% Norman Christ
\subsection{Two photon exchange contribution to $K_L\to\mu^+\mu^-$}
\begin{itemize}
    \item{\bf Motivation:} The $\Delta S=1$ neutral-current rare decay $K_L\to\mu^+\mu^-$ 
    offers an order $G_F^2$ test of the standard model in a process involving the 
    exchange of two $W$ bosons or a $W$ and a $Z$ boson.  However, a background, 
    two-photon exchange process of order $\alpha_{\mathrm{EM}}^2G_F$ contributes 
    at a similar strength to this decay and therefore must be computed if a standard 
    model prediction is to be compared with experiment.
    \item{\bf Long term goal:} Calculate this two-photon exchange contribution to 5\% 
    accuracy so that the standard model prediction for the important $G_F^2$ process 
    can be compared to experiment at the 5-10\% precision comparable to the accuracy 
    allowed by the current experimental branching ratio for $K_L\to\mu^+\mu^-$: 
    BR$(K_L\to\mu^+\mu^-) = 6.84\pm0.11\times 10^{-9}$.
    \item{\bf Method:}  This process is similar to the hadronic light-by-light scattering 
    contribution to the muon.  However, new methods are needed because the 
    real-time, complex decay amplitude cannot be directly evaluated in Euclidean 
    space.  However, a well-controlled lattice QCD formulation of this calculation 
    has been developed in which the largest uncontrolled error arises from the 
    $\pi\pi\gamma$ intermediate state whose contribution is expected to be no 
    more than a few percent.
\item{\bf Timeline:} \hfill [last updated May 2023]
\begin{itemize}
    \item{\bf 2000} Accurate measurement of $K_L\to\mu^+\mu^-$ branching 
    ratio~\cite{E871:2000wvm}.
    \item{\bf 2018-2019} Lattice QCD method devised~\cite{Christ:2020bzb} and 
    first results obtained for the simpler $\pi^0\to e^+ e^-$ 
    decay~\cite{Christ:2020dae, Christ:2022rho}.
    \item{\bf 2019-2022} Calculation extended to the $\Delta S=1$ process 
    $K_L\to\gamma\gamma$ and first results obtained on a single gauge 
    ensemble including only connected graphs with physical quark masses 
    but for a relatively large lattice spacing.~\cite{Zhao:2022pbs, Zhao:2022njd}
    \item{\bf 2022-2024} First calculation of the two-photon exchange contribution 
    to $K_L\to\mu^+ \mu^-$ on a single gauge ensemble including only connected 
    graphs with physical quark masses but for a relatively large lattice spacing.
    \item{\bf 2025-2028} Extend the calculation to multiple lattice spacings and include 
    disconnected graphs.  Achieve 10\% precision.
    \item{\bf 2028-2031} Employ increased statistics and improved methods to 
    achieve the targeted 5\% precision.
\end{itemize}
\end{itemize}


% Luchang Jin/Norman Christ
\subsection{QED correction to pion, kaon, and Omega baryon masses}
\begin{itemize}
    \item{\bf Motivation:} Determine the quark masses and lattice spacing with QED corrections. Provide the basis for the QED correction for other quantities, such as hadronic vacuum polarization contribution to muon $g-2$ and meson (semi-)leptonic decay.
    \item{\bf Long term goal:} Calculate pion, kaon, and Omega baryon mass with QED corrections to per-mil level precision.
    \item{\bf Method:}
       Infinite volume reconstruction (IVR) method \cite{Feng:2018qpx} to eliminate
      the $1/L,~1/L^2,~\cdots$ finite volume effects due to massless photon interactions.
\item{\bf Timeline:} \hfill [last updated March 2024]
\begin{itemize}
    \item{\bf 2019} IVR method introduced
    \item{\bf 2020-2022} Pion mass splitting calculated with the IVR method~\cite{Feng:2021zek}.
    \item{\bf 2023-2026} Calculate the QED corrections to the mass of pion, kaon, and Omega baryon with the IVR method to sub-percent precision relative to the hadron mass.
\end{itemize}
\end{itemize}

% Luchang Jin/Norman Christ
\subsection{QED correction to pion and kaon (semi-)leptonic decay}
\begin{itemize}
    \item{\bf Motivation:} Determine the CKM matrix elements $|V_{ud}|,~|V_{us}|$ from pion and kaon (semi-)leptonic decays.
    \item{\bf Long term goal:} Calculate pion and kaon decay width with QED corrections to sub-percent level accuracy, matching or exceeding the current experimental accuracy.
    \item{\bf Method:}
      Infinite volume reconstruction (IVR) method \cite{Christ:2023lcc, Christ:2024xzj} used to eliminate
      the $1/L,~1/L^2,\,\dots$ finite volume effects due to massless photon interactions.
\item{\bf Timeline:} \hfill [last updated March 2024]
\begin{itemize}
    \item{\bf 2019} IVR method introduced.
    \item{\bf 2020-2023} Develop application of IVR method to meson (semi-)leptonic decay.
    \item{\bf 2023-2026} Calculate the QED corrections to the leptonic decay width to an accuracy comparable to or better than the current experimental accuracy.
    \item{\bf 2027-2030} Calculate the QED corrections to the (semi-)leptonic decay width.
\end{itemize}
\end{itemize}


% Anna Hasenfratz
%INTENSITY FRONTIER  ( QCD strong coupling, as defined at the Z-pole)
\subsection{The QCD $\Lambda$ parameter and the  strong coupling constant at the Z-pole}
\begin{itemize}
	\item{\bf Motivation:}  The most precise lattice determination of
          the QCD strong coupling $\alpha_s$  comes from the European
          Alpha collaboration but  it should require independent and
          important  cross-validation. Two  USQCD  groups are working
          toward  this goal. 
	\item{\bf Long term goal:} The QCD strong coupling $\alpha_s$, as
          defined at the Z-pole, is equivalent to the long term goal
          to determine $\Lambda_{\overline{MS}}$  in physical units
          with very high precision. Credible and high precision
          cross-validation  of this goal  is motivated by the
          emergence of  the required new lattice  technology for the
          long-term goal. 
	\item{\bf Method:}  The application of gradient flow based
          scale-dependent renormalization of
          $\alpha_s(\mu=1/\sqrt{8t})$ at flow time scale $t$ on
          lattices extended to  infinite space-time provides a new
          technology, originally developed for beyond Standard Model studies. The application of the method in QCD shows the constant cross-fertilization  between the different subfields  of lattice studies. 
\item{\bf Timeline:} \hfill [last updated May 2024]
\begin{itemize}
\item{\bf 2016-2020} A new paradigm to calculate the nonperturbative renormalization group $\beta$ function using the gradient flow renormalization scheme was developed by several USQCD groups. The continuous $\beta$ function (CBF) method is particularly promising as it can be applied both in the weak coupling deconfined and strong coupling confining phases. The approach was tested in two-flavor QCD and in multiflavor QCD with ten and twelve massless flavors.
\item{\bf 2022-2023} In the {\rm SU}(3) Yang-Mills limit of quenched
  QCD it was shown by two USQCD groups that the new method is a
  competitive high-precision match to the earlier method of the Alpha
  collaboration~\cite{Hasenfratz:2023bok,Wong:2023jvr}. The combined
  high accuracy is in significant tension with any other lattice
  method.
\item{\bf 2024-2025} A large number of  lattice gauge configuration have been developed to calculate the renormalized gauge coupling and the beta function on the gradient flow to determine the QCD coupling at the Z-boson pole with high precision. This requires to connect the fundamental scale parameter of the running coupling to a physical scale, like the pion decay constant F in the chiral limit of massless quarks. The analysis targets QCD with three flavors.          
\end{itemize}
\end{itemize}



\section{Hadron spectroscopy}\label{sec:hadspec}

% Robert Edwards
\subsection{Exotic light quark mesons}
\begin{itemize}
    \item{\bf Motivation:} Observation of putative exotic $J^{PC}$ bosonic
      states. Focus of GlueX@JLab, CLAS12@JLab, COMPASS@LHC.Kaon PID
      upgrade of GlueX and recommissioning starting 2020. Phase II
      including JLab Eta Factory underway through 2025. Possible Phase
      III run.  
    \item{\bf Long term goal:} Determine the mass and decay modes of
      putative light quark hybrid and exotic mesons. Extract resonance
      parameters of meson and baryon spectrum. Guide experimental
      searches with predictions of decay couplings.
    \item{\bf Method:} Resonance spectrum extracted from scattering
      amplitudes computed from finite-volume energy spectrum. Cost
      driven by annihilation quark lines computed on many
      time-slices and computation of hadronic two-point functions
      featuring a large multi-hadron operator basis utilizing the variational method. 
\item{\bf Timeline:} \hfill [last updated May 2024]
\begin{itemize}
    \item{\bf 2013} Isovector and isoscalar light quark meson and baryon
      spectrum computed with restriction to single particle basis~\cite{Dudek:2013yja}. Results featured in PDG.
    \item{\bf 2015} First computation of resonance parameters from coupled-channel scattering
      amplitudes of $\pi\pi/\bar{K}K$ \cite{Wilson:2015dqa}.
    \item{\bf 2019} Phenomenological extraction of $\pi_1$ resonance from 
      parameters from partial waves of $\eta(')\pi$ measured by COMPASS~\cite{JPAC:2018zyd}.
    \item{\bf 2022} First determination of full three-body relativistic scattering amplitude~\cite{Hansen:2020otl}
    \item{\bf 2022} Prediction for light-quark isovector $J^{PC}=1^{-+}$ published~\cite{Woss:2020ayi}.
    \item{\bf 2023} Upgrade of GlueX forward calorimeter.
    \item{\bf 2025} Putative light-quark hybrid meson multiplet resonance parameters.
    \item{\bf 2025} GlueX Phase II completed. Begin analysis of runs I and II. 
    \item{\bf 2025} Possible Phase III GlueX run.
    \item{\bf 2027} GlueX results based on combined data through run II.
\end{itemize}
\end{itemize}



% Robert Edwards
\subsection{Internal structure of resonant states}
\begin{itemize}
    \item{\bf Motivation:} The internal structure of QCD resonant states
      is poorly understood. Their study is a focus of GlueX@JLab,
      CLAS12 MesonX@JLab, LHCb@CERN, BESIII@Beijing. 
    \item{\bf Long term goal:} Use electromagnetic probes to reveal the internal structure of resonant
      states, revealing potential multi-particle
      configurations through computation of physically relevant
      processes, like transition form-factors, or more complicated
      structures such as partonic content.
    \item{\bf Method:} Combines QCD spectroscopy program and hadronic
      structure programs. Infinite volume current matrix elements
      extracted from scattering amplitudes.
\item{\bf Timeline:} \hfill [last updated May 2024]
\begin{itemize}
    \item{\bf 2015} First computation of $\gamma\pi\rightarrow\pi\pi$ form-factor~\cite{Briceno:2015dca}. 
    \item{\bf 2022} First computation of $\gamma K\rightarrow K\pi$
      form-factor~\cite{Radhakrishnan:2022ubg} featuring multi-partial
      wave form-factors.
    \item{\bf 2024} Energy dependence of $K\bar{K}/\pi\pi$ to vacuum transition
    \item{\bf 2024} $J/\psi\rightarrow\gamma\eta$ radiative decays with  comparison to BESIII
    \item{\bf 2025} Multi-channel nucleon transition form-factors of vector and axial currents.
    \item{\bf 2025} Radiative transition form-factors of exotic isovector $\pi_1$.
    \item{\bf 2026} GlueX electromagnetic branching fraction rates for exotic meson states.
\end{itemize}
\end{itemize}

% Robert Edwards
\subsection{Light and strange quark baryon spectroscopy}
\begin{itemize}
    \item{\bf Motivation:} Experimental $N^\ast$ program in CLAS12@JLab,
      and strange quark hyperon resonances in CLAS12@JLab. Early LQCD
      spectroscopy calculations suggest rich spectrum of states - most
      of which have not been experimentally observed.
    \item{\bf Long term goal:} Fully resonance calculations of decay
      couplings of possible states, including potential hybrid baryon
      states identified through transition form-factors.
    \item{\bf Method:} 
  \item{\bf Timeline:} \hfill [last updated May 2024]
\begin{itemize}
    \item{\bf 2012} Initial calculations of light and strange quark baryon
      spectrum using single particle operators~\cite{Edwards:2011jj,Dudek:2012ag,Edwards:2012fx}.
    \item{\bf 2024} Coupled-channel low lying negative parity $\Delta$ resonances.
%    \item{\bf 2025} Coupled-channel negative parity $\Lambda$ baryons.
    \item{\bf 2025} Potential charm-light quark baryon resonances.
\end{itemize}
\end{itemize}





\section{Hadron structure}\label{sec:hadstruct}

\subsection{Electromagnetic Form Factors}\label{sec:vecff}

% Rajan Gupta
\subsubsection{Nucleon charges}\label{sec:nuccharges}
\begin{itemize}
    \item{\bf Motivation:} Nucleon charges $g_{A,S,T}$ arise in many
      low-energy description of nucleons. Flavor diagonal axial
      charges give the contributions of each quark flavor to the
      nucleon spin; tensor charges give the contribution of quark EDM
      to the nucleon EDM; and scalar charges give the pion-nucleon sigma term and strangeness content of the nucleon. 
      Isovector charges give the axial charge $g_A$, a fundamental low-energy constant. The scalar and tensor 
      charges probe novel scalar and tensor interactions at the TeV scale and the tensor charge is also 
      measured in transversity measurements at JLAB. From the matrix elements of 1-link operators, we extract 
      the first moment of distributions, namely momentum fraction, and helicity and transversity moments. These 
      results compare favorably with their extractions from PDFs.
    \item{\bf Long term goal:} To calculate each of these with sub-percent accuracy. 
    \item{\bf Method:} All these require the calculation of connected and disconnected three-point
      functions comprising the insertion of appropriate quark bilinear operators between 
      nucleon source and sink operators. During analysis, remove the
      excited state contributions from these correlation functions to
      get ground state matrix elements that are then decomposed into
      the desired charges. The calculation is repeated at
      multiple values of lattice spacing and pion mass and the results
      extrapolated to the physical point.
\item{\bf Timeline:} \hfill [last updated May 2023]
\begin{itemize}
    \item{\bf 2011--2019} Lattice results reach the robustness standard to be reviewed by FLAG and average values published.
    \item{\bf 2019-2023} Steady improvement in precision to reach few  percent accuracy and get control over excited states. 
      Possible resolution of the discrepancy between lattice estimate of the the pion-nucleon sigma term between lattice QCD and 
      dispersive analysis published by the PNDME collaboration. 
    \item{\bf 2023-2025} Continue to improve accuracy. 
    \item{\bf 2030} Much higher precision analysis of neutron decay that
      would push the search for novel scalar and tensor interactions to the $10^{-4}$ level. 
\end{itemize}
\end{itemize}

% Rajan Gupta
\subsubsection{Nucleon electromagnetic form factors}
\begin{itemize}
    \item{\bf Motivation:} The electromagnetic form factors of the nucleons are well measured experimentally from electron 
      scattering off nuclei.  There is still uncertainty in the calculation of the charge radius of the proton, i.e., 
      a difference from scattering experiments versus muonic hydrogen.  These data are far more precise compared to 
      lattice results and for the foreseeable future will provide a benchmark against which to compare lattice results to 
      and validate lattice methodology. 
    \item{\bf Long term goal:} To determine the electromagnetic form factors of the nucleons to match experimental precision. 
    \item{\bf Method:} First calculate three point functions comprising the insertion of the electromagnetic current
      with all allowed lattice momentum insertion between the nucleon source and sink operators. During analysis, remove 
      the excited state contributions from these correlation functions to get ground state matrix elements that are 
      then decomposed into the desired form factors. The calculation is repeated at multiple values of lattice spacing and 
      pion mass and the results extrapolated to the physical point. 
\item{\bf Timeline:} \hfill [last updated May 2023]
\begin{itemize}
    \item{\bf 2011--2021} Lattice results showed deviations from the Kelly parameterization of experimental data. 
    \item{\bf 2021} NME results agree with Kelly parameterization of experimental data.
    \item{\bf 2021-2023} Calculations by various groups still show an uncertainty band of $\approx 10\%$.
    \item{\bf 2023-2025} NME and PNDME collaborations to reduce the uncertainty band to 3--5\%
    \item{\bf 2030} EIC expected to start taking data.
\end{itemize}
\end{itemize}

% Sergey Syritsyn
\subsubsection{Calculation of high-momentum nucleon form factors}
\begin{itemize}
    \item{\bf Motivation:} Any data on nucleon form factors in the range of momentum transfer 5-20 GeV$^2$ will be important for understanding the transition between perturbative and nonperturbative QCD. Such data will also be illuminating for qualitative pictures of nucleons in terms of gluon-dressed constituents or diquarks.
    \item{\bf Long term goal:} Ab initio calculation of such form factors using QCD on a lattice, done with fully controlled systematic errors, is necessary to fill the gap between theory and ongoing experiments. Comparison to experimental results will also reveal and help address any systematic uncertainties in calculations with high-momentum hadron states on a lattice such as calculations of PDFs/GPDs using LaMET and similar approaches.
    \item{\bf Method:} We employ standard lattice techniques together with some methods (e.g., momentum-boosted states) to improve overlap with large-momentum nucleon ground state. We use Breit frame kinematics to reach maximal possible momentum transfer.
\item{\bf Timeline:} \hfill [last updated May 2023]
\begin{itemize}
    \item{\bf 2016-2023} We have performed calculations of electric and magnetic form factors at three values of lattice spacing and pion masses down to 170 MeV. We have found qualitative agreement in the dependence of the proton form factor ratios $G_{Ep}/G_{Mp}$ and $F_{1p}/F_{2p}$. Some agreement is also evident in the neutron case for $G_{En}/G_{Mn}$. While this agreement is reassuring, the values of the form factors themselves overshoot experiment by factor 2-2.5 for momenta $Q^2$ above 4 GeV$^2$. This is probably due to large contributions of the excited states, since we observe very little dependence on the lattice spacing and the pion mass.
    \item{\bf 2023-onward} We plan to increase the statistical precision and extend the range of nucleon correlators in order to improve the control over excited states. In addition, we plan to study the so-called disconnected diagrams (contributions to the isoscalar $(p+n)$ channel) and the leading-order $O(a)$ discretization corrections.
    \item{\bf Relevant experiments} have been performed at CEBAF (JLab@12 GeV) to measure electromagnetic form factors up to 18 GeV$^2$. Results for the magnetic form factor of the proton have been published.
\end{itemize}
\end{itemize}

\subsection{Axial Form Factors}\label{sec:axff}

% Rajan Gupta
\subsubsection{NME, PNDME Collaborations}
\begin{itemize}
    \item{\bf Motivation:} The axial form factors of the nucleons are not well measured experimentally and are
      needed to few percent accuracy to calculate the neutrino-nucleus cross-section for experiments such as DUNE, T2K, etc., to 
      reach their science goals. Large scale simulations of lattice QCD are the current 
      best method to determine these directly from QCD and the methodology for the 
      calculations is mature. 
    \item{\bf Long term goal:} To determine the axial form factors of the nucleons to a few percent 
    \item{\bf Method:} First calculate three point functions comprising the insertion of the axial vector and pseudoscalar currents 
      with allowed lattice  momentum insertion between the nucleon source and sink operators. During analysis, remove 
      the excited state contributions from these correlation functions to get ground state matrix elements that are 
      then decomposed into the desired form factors. The calculation is repeated at multiple values of lattice spacing and 
      pion mass and the results extrapolated to the physical point. 
\item{\bf Timeline:} \hfill [last updated May 2023]
\begin{itemize}
    \item{\bf 2017} Demonstration that the standard method of analysis gives form factors that do not satisfy PCAC.
    \item{\bf 2019} Cause of failure identified to be the contributions of multihadron ($N\pi, N\pi\pi, \ldots$) states 
      and a data driven method to remove them identified.
    \item{\bf 2021-2023} Calculations done by a number of lattice collaborations that agree within $1\sigma$ and give results
      with an uncertainty band of $\approx 10\%$.
    \item{\bf 2023-2025} NME and PNDME collaborations to reduce the uncertainty band to 3--5\%
    \item{\bf 2030} DUNE expected to start taking data.
\end{itemize}
\end{itemize}

\subsubsection{Fermilab Lattice Collaboration}
\begin{itemize}
   \item{\bf Motivation:} Explore feasibility of nucleon-matrix-element calculations with staggered valence quarks, motivated by the neutrino-scattering experiments listed in the previous item.
   \item{\bf Long term goal:} Axial charge $g_A$ and axial form factor with the HISQ action for valence quarks, using MILC's $2+1+1$-flavor ensembles.
   \item{\bf Method:} Staggered fermions have a more complicated symmetry structure than Wilson-like quarks (including domain-wall, twisted-mass, clover, \ldots), making nucleon operators more complicated to construct.  While we have solved this problem for the needed two- and three-point calculations, the approach does not turn out to enjoy the computational savings that staggered fermions provide for mesons.
    \item{\bf Computing:} Work on ensembles with $a\approx0.15$ and 0.12~fm were carried out on USQCD resources (FNAL~LQ1, BNL~Sklake, and BNL~KNL), while work on ensembles with $a\approx0.09$ and 0.06~fm were carried out on LCF resources (NERSC Cori, XSEDE Stampede2, ALCF Theta).
\item{\bf Timeline:} \hfill [last updated May 2023]
\begin{itemize}
   \item{\bf 2019} Calculation of the nucleon mass at three spacings with physical quark mass~\cite{Lin:2019pia}; at the time (and possibly still), this was the most precise calculation of the nucleon mass.
   \item{\bf 2021} Demonstration calculation of the axial charge $g_A$ on the physical-quark-mass HISQ ensemble with $a\approx0.12$~fm~\cite{Lin:2020wko}.
   \item{\bf 2023} Calculation of $g_A$ at four spacings ($a\approx0.15$--0.06~fm) with physical quark mass, with update to the nucleon mass.
   \item{\bf 2023} Demonstration calculation of the axial charge $g_A$ on the physical-quark-mass HISQ ensemble with $a\approx0.12$~fm~\cite{Lin:2020wko}.
\item{\bf Outlook:} The all-HISQ approach to baryons seems not to be cost-effective enough to justify the overhead in coding and analysis.
\end{itemize}
\end{itemize}



\subsection{Nucleon Electric Dipole Moment (nEDM)}\label{sec:nEDM}

% Rajan Gupta
\subsubsection{Contributions of CP violating operators to nucleon EDM}
\begin{itemize}
    \item{\bf Motivation:} The CP violation in the CKM quark mixing is too
      small to generate the observed baryon asymmetry in the observed
      universe.  BSM models have new sources of CP violation, each of
      which contribute to nEDM. These interactions and their couplings
      in BSM theories are, using tools of effective field theories,
      written in terms of low energy operators composed of quark and
      gluon fields. The matrix elements of these effective operators
      are only known within a factor of 10 uncertainty.  Lattice QCD
      provides the best method to determine them with control over all
      systematics. Knowing these matrix elements, one can use the bound
      [eventually value] of nEDM to constrain the CP violating
      couplings in various BSM theories.  These viable BSM models can
      then be analyzed to determine if they are further consistent
      with the generation of the observed baryon asymmetry.
    \item{\bf Long term goal:} To calculate the matrix elements of all low
      energy effective operators of dimension six 
      and smaller that contribute to the nucleon electric dipole moment. 
    \item{\bf Method:} This requires calculating three- and four- point
      functions comprising the insertion of the electromagnetic
      current with finite CP violating operators with finite momentum
      insertion between the nucleon source and sink operators to be
      able to take the zero momentum limit.  During analysis, remove
      the excited state contributions from these correlation functions
      to get ground state matrix elements that are then decomposed
      into the desired form factors of which $F_3$ is the desired CP
      violating one from which $d_n = F_3(0)/2 M_N \epsilon$, where
      $\epsilon $ is the coupling strength of the operator. Repeat the
      calculation at multiple values of lattice spacing and pion mass
      and extrapolate the results to the physical point.
\item{\bf Timeline:} \hfill [last updated May 2023]
\begin{itemize}
    \item{\bf Pre 2017} The first lattice calculations set up the
      framework but did not include a phase in the definition of the
      nucleon spinor, which lead to wrong results.  
    \item{\bf 2017-2023} Focus on error reduction in the extraction of
      $F_3$ parameterization of experimental data. Understand and remove 
      the contributions of multihadron ($N\pi,
      N\pi\pi, \ldots$) states. 
    \item{\bf 2021-2023} Calculations from a number of collaboration for
      the Theta term with different levels of control over signal and
      systematics.  
    \item{\bf 2023-2025} To get a robust signal at near physical quark masses. 
    \item{\bf 2030} Expect a factor of 10 reduction in the upper bound on
      neutron EDM and maybe a value.  
\end{itemize}
\end{itemize}

% Sergey Syritsyn
\subsubsection{Quark-gluon CP violating contributions}
\begin{itemize}
    \item{\bf Motivation:} Neutron and proton electric dipole moments
      (EDMs) are the most precise probes of CP violation in QCD and
      beyond-the SM physics. The latter is important for finding
      plausible scenarios of baryogenesis and understanding the origin
      of the nuclear matter. Knowing the magnitude of neutron EDM
      induced by different sources of CP violation is crucial to using
      experimental bounds to constrain new types of CP-violating
      interactions. 
    \item{\bf Long term goal} is finding contributions to nucleon EDMs of
      effective quark-gluon CPv operators that may be induced by new
      physics. These effective CPv interactions are quark EDMs, quark
      and gluon chromo-EDMs, QCD theta-term, and 4-quark
      interactions. 
    \item{\bf Method:} We use the background-field method in which the
      energy (mass) of a nucleon is shifted due to induced EDM. This
      method may have comparative advantage compared to computing
      electric dipole form factors that have to be extrapolated to the
      forward limit. In addition, it may be simpler to implement in
      the case of four-quark operators. 
\item{\bf Timeline:} \hfill [last updated May 2023]
\begin{itemize}
    \item{\bf 2016-2023} We have successfully demonstrated that the
      background field method is at least comparable in precision to
      the traditional method of computing neutron EDM induced by the
      QCD theta-term.We have implemented sophisticated sampling
      techniques combining variable precision and low-lying eigenmode
      approximation of the Dirac equation. We have performed
      calculation of the neutron EDM using chirally-symmetric
      fermions. 
    \item{\bf 2023-onward} We plan to extend our calculations of
      theta-QCD-induced neutron EDM to the physical point, where the
      low-eigenmode approximation is expected to be even more
      efficient. We also plan to start calculations involving 4-quark
      CPv operators. 
    \item{\bf Experiments} are expected to improve bounds on neutron EDMs
      within the current decade. It is particularly important to
      improve our theoretical knowledge of neutron EDM contributions
      in the same time frame in order to constrain proposed models of
      novel CP-violating interactions and baryogenesis scenarios. 
\end{itemize}
\end{itemize}


\subsection{Hadron tomography}

% Xiang Gao
\subsubsection{Internal structure of Goldstone mesons and nucleon}
\begin{itemize}
    \item{\bf Motivation:} Studying the 3D-structure of hadrons and understanding the partonic origin of hard exclusive processes are among the goals of the Jefferson Lab 12 GeV program and the future Electron-Ion Collider (EIC). The pion and kaon, as the pseudo-Goldstone modes of QCD, play a critical role in understanding the physics of asymptotic freedom and chiral symmetry-breaking in QCD from the fundamental quark-gluon interactions.
    
    \item{\bf Long term goal:} First principles calculations of the pion and kaon distribution amplitudes and form factors with large momentum transfer to explore the transition from non-perturbative QCD to perturbative QCD. Precision calculations of the pion valence quark PDFs and GPDs, providing insights into the quark momentum fractions and 3-dimensional distributions in the pseudo-Goldstone bosons. Exploring the spin-dependent PDFs and 3-dimensional structure of nucleon.
    
    \item{\bf Method:} Calculation of pion/kaon form factors in the Breit frame with optimized quark smearing to achieve large momentum transfers. Computation of the quasi-DA, quasi-PDF, quasi-GPD, and quasi-TMD matrix elements within large-momentum hadron states to extract the light-cone parton distributions using large momentum effective theory and short distance factorization.
      
\item{\bf Timeline:} \hfill [last updated May 2024]
\begin{itemize}
    \item{\bf 2019} Calculation of the pion valence quark distribution with a 300 MeV pion mass from a fine lattice within the LaMET framework.~\cite{Izubuchi:2019lyk}
    \item{\bf 2020} Calculation of the pion valence quark distribution with a continuum extrapolation using the LaMET and short distance factorization frameworks.~\cite{Gao:2020ito}
    \item{\bf 2021} Calculation of the structural differences between the pion and its radial excitation.~\cite{Gao:2021hvs}
    \item{\bf 2021} Calculation of the pion form factors and charge radius with small momentum transfer at the physical point.~\cite{Gao:2021xsm}
    \item{\bf 2022} Calculation of the pion distribution amplitude at the physical point within the short distance factorization framework.~\cite{Gao:2022vyh}
    \item{\bf 2022} Calculation of the Bjorken-x dependence of the pion valence PDF with controlled systematics.~\cite{Gao:2021dbh}
    \item{\bf 2022} Calculation of the pion valence PDF with a continuum extrapolation, physical quark masses and NNLO matching.~\cite{Gao:2022iex}
    \item{\bf 2022} Calculation of the Unpolarized PDF of nucleon.~\cite{Gao:2022uhg}
    \item{\bf 2023} Calculation of the Transversity PDF of nucleon.~\cite{Gao:2023ktu}
    \item{\bf 2024} Calculation of the kaon distribution amplitude at the physical point.
    \item{\bf 2024} First calculation of the pion form factors at the physical point with large momentum transfer.
    \item{\bf 2024} First calculation of the kaon form factors at the physical point with large momentum transfer.
    \item{\bf 2025} Calculation of pion TMDs with a 300 MeV pion mass.
    \item{\bf 2025} Calculation of pion GPDs with a 300 MeV pion mass.
    \item{\bf 2027} Calculation of pion GPDs at the physical point.
\end{itemize}
\end{itemize}


    
% David Richards/Joe Karpie
\subsubsection{Internal structure of mesons and nucleons}\label{sec:meshadstruct}
\begin{itemize}
  \item{\bf Motivation:} A quantitative understanding of the internal
     structure of hadrons, including the distribution of charge, spin
     and mass, in terms of the quarks and gluons of QCD is a central
     goal of nuclear physics, and key to high-energy physics through
     exposing the Standard Model and Beyond-the-Standard Model
     interactions of the quarks and gluons within them.  Flagship
     experiments at JLab\@12 GeV focused on isovector GPDs, and
     understanding the gluonic contributions to hadron structure a
     emblematic problem at the future EIC.
   \item{\bf Long term goal:} First principles calculations of the key
     measures of one-dimensional and three-dimensional hadron
     structure encapsulated in the Parton Distribution Functions (PDFs),
     Generalized Parton Distribution Functions (GPDs) and
     Transverse-Momentum-Dependent Functions (TMDs) as well as studying their evolution kernels from first principles.  Performing
     computations of a precision that can both confront experiment,
     and complement experiment in global analysis.
   \item{\bf Method:} Computation of the three-point matrix elements of quark- and
     gluonic operators, and their analysis to provide
     Bjorken-$x$-dependent distributions within the
     short-distance-factorization/pseudo-PDF framework.  Exploitation
     of the ``distillation'' both to fully sample lattices, and
     provide control over excited-state contributions.
\item{\bf Timeline:} \hfill [last updated May 2024]
\begin{itemize}
   \item{\bf 2017} First calculation of the nucleon unpolarized PDF within
     the pseudo-PDF framework~\cite{Orginos:2017kos}.
   \item{\bf 2020} Calculation of unpolarized nucleon PDF at close-to-physical
     quark masses~\cite{Joo:2020spy}.
   \item{\bf 2021} Calculation of the Unpolarized Gluon Distribution in
     the Nucleon~\cite{HadStruc:2021wmh}.
   \item{\bf 2022} Calculation of the gluon helicity distribution in
     the nucleon~\cite{HadStruc:2022yaw}
   \item{\bf 2022} Combined global analysis of lattice QCD and
     experimental data to compute the Pion PDF~\cite{JeffersonLabAngularMomentumJAM:2022aix}.
    \item{\bf 2023} Combined global analysis of lattice QCD and experimental data for gluon spin structure of the nucleon~\cite{Karpie:2023nyg}
    \item{\bf 2023} First non-perturbative study of evolution of nucleon pseudo-distribution~\cite{Dutrieux:2023zpy}
   \item{\bf 2024} Calculation of the isosinglet unpolarized and
     polarized nucleon PDF.
   \item{\bf 2024} First Calculation of the nucleon Generalized Parton
     Distributions within the pseudo-PDF framework.
   \item{\bf 2027} E12-06-119 ``Deeply Virtual Compton Scattering with
     CLAS at 11 GeV''
   \item{\bf 2027} SOLID detector at Jefferson Lab
   \item{\bf 2030} Electron-Ion Collider at BNL.
\end{itemize}
\end{itemize}


% Michael Engelhardt    
\subsubsection{Transverse momentum-dependent parton distribution
(TMD) observables.}
\begin{itemize}
\item{\bf Motivation:} TMDs constitute one of the pillars of the description
of hadron structure; they encode the three-dimensional distribution of
momenta among the partons inside a hadron. They enter the cross sections
of processes such as semi-inclusive deep inelastic scattering (SIDIS) and
the Drell-Yan process. Their determination constitutes a focus of both the
current JLab 12 Gev program as well as the experimental program at the EIC.
\item{\bf Long term goal:} Calculation of TMD observables to
complement and augment phenomenological analyses of experimental data,
and to help guide follow-on experimental campaigns.
\item{\bf Method:} Lattice QCD evaluation of hadronic matrix elements of
bilocal operators containing staple-shaped gauge connections. Extraction
of invariant amplitudes that determine ratios of Fourier-transformed TMDs.
\item{\bf Timeline:} \hfill [last updated May 2024]
\begin{itemize}
\item{\bf 2011} First calculation of Sivers and Boer-Mulders observables.
\item{\bf 2015} Extrapolation of a TMD observable to the physical, infinite
rapidity difference limit.
\item{\bf 2017} Investigation of universality of lattice TMD observables.
\item{\bf 2019} Evaluation of TMD observables at the physical pion mass.
\item{\bf 2022} Completion of leading-twist set of nucleon TMD observables by
including longitudinal nucleon polarization.
\item{\bf 2024} Evaluation of gauge connection structures beyond staple shape,
relevant for processes beyond SIDIS and Drell-Yan.
\item{\bf 2024} Evaluation of longitudinal momentum fraction $x$-dependence
of the Sivers shift.
\item{\bf 2027} Inclusion of Lattice TMD observables with controlled systematic
uncertainties in global analyses of experimental data.
\end{itemize}
\end{itemize}


% Yong Zhao
\subsubsection{Collins-Soper kernel for TMD evolution}
\begin{itemize}
    \item{\bf Motivation:} The Collins-Soper kernel relates the
      transverse-momentum-dependent distributions (TMDs) of hadrons at
      different energy scales, which is a crucial non-perturbative
      input for global analyses to understand the 3D structure of the
      proton.
    \item{\bf Long term goal:} Calculate the Collins-Soper kernel at
      physical pion mass with a $\sim 10\%$ precision, which can be
      compared to the experimental results from the Electron-Ion
      Collider.
    \item{\bf Method:} Lattice calculation and renormalization of the
      matrix elements of quasi-TMD correlators in a highly boosted
      hadron state. The quasi-TMD correlators are defined from quark
      and gluon bilinear operators with a staple-shaped Wilson line
      extending to the longitudinal direction as the hadron
      momentum. At large hadron momentum, the quasi-TMD can be
      factorized into the physical TMD, and the anomalous dimension of
      their momentum evolution can be perturbatively matched onto the
      Collins-Soper kernel. Apart from the forward matrix elements for
      quasi-TMDs, we can also use the TMD wave function, which is a
      vacuum to hadron amplitude of the quasi-TMD correlator, to
      extract the Collins-Soper kernel.
\item{\bf Timeline:} \hfill [last updated May 2024]
\begin{itemize}
    \item{\bf 2019} Study of the renormalization and mixing of the
      staple-shaped quasi-TMD correlator on the
      lattice~\cite{Shanahan:2019zcq}.
    \item{\bf 2020} First exploratory calculation of the quark
      Collins-Soper kernel on a quenched lattice ensemble from the
      quasi-TMDs~\cite{Shanahan:2020zxr}.
    \item{\bf 2021} First lattice calculation of the quark Collins-Soper
      kernel with dynamical fermions at unphysical pion mass of around
      500 MeV from the quasi-TMDs~\cite{Shanahan:2021tst}.
    \item{\bf 2022-2023} Lattice calculation of the quark
      Collins-Soper kernel at the physical point from the quasi-TMD
      wave functions~\cite{Avkhadiev:2023poz}.
    \item{\bf 2024} Continuum extrapolation of the quark Collins-Soper kernel at the physical point, with a precision that can distinguish phenomenological models in the non-perturbative region~\cite{Avkhadiev:2024mgd}.
    \item{\bf 2024-2026} Develop and implement the method to calculate the gluon Collins-Soper kernel.
    \item{\bf 2024-2026} Lattice calculation of the quark TMD soft factor.
\end{itemize}
\end{itemize}

% Michael Engelhardt    
\subsubsection{Direct evaluation of parton orbital angular momentum (OAM) in
the proton}
\begin{itemize}
\item{\bf Motivation:} The proton spin puzzle is a touchstone of our
understanding of proton structure. Compared to the parton spin
contributions, the parton OAM contributions are less straightforward
to access directly, requiring the evaluation of either generalized
transverse momentum-dependent parton distribution (GTMD) observables
or twist-three generalized parton distribution (GPD) observables.
Lattice QCD provides an avenue to obtain these observables.
\item{\bf Long term goal:} Evaluation of GTMD and twist-three GPD observables
to determine parton OAM in the proton.
\item{\bf Method:} Lattice QCD evaluation of hadronic matrix elements of
bilocal operators containing both straight and staple-shaped gauge
connections; these determine the Ji and the Jaffe-Manohar definitions
of parton OAM, respectively.
\item{\bf Timeline:} \hfill [last updated May 2024]
\begin{itemize}
\item{\bf 2016} First evaluation of both Ji and Jaffe-Manohar quark OAM from
GTMD observables.
\item{\bf 2018} Control of systematic uncertainties in the momentum transfer
dependence of the relevant GTMD through a direct derivative method,
leading to quark angular momentum sum rule being satisfied.
\item{\bf 2021} Evaluation of quark spin-orbit correlations in the proton.
\item{\bf 2023} Evaluation of quark OAM in the proton via twist-three GPDs.
\item{\bf 2024} Enhancing control of systematic uncertainties in Lattice QCD
evaluations of quark OAM; quark mass dependence, excited state effects.
\item{\bf 2027} Evaluation of gluon OAM.
\end{itemize}
\end{itemize}



% Bigeng Wang
\subsubsection{Gravitational form factors of the energy momentum tensor}
\begin{itemize}
    \item{\bf Motivation:} The gravitational form factors (GFF) of the energy-momentum tensor (EMT) encompass the issues of spin and mass decompositions. The hadron mass can be decomposed into the trace anomaly and sigma terms.
    \item{\bf Long term goal:} Provide Standard Model predictions of the gravitational form factors of hadron energy-momentum tensor. These results can be confirmed by the experimental GPD and photo-production of J/$\Psi$ at the threshold.
    \item{\bf Method:} Lattice calculations with grid source propagators and low-mode substitution (LMS) of hadronic two- and three-point functions using domain-wall fermion ensembles and valence overlap fermions. The form factors are measured at several lattice spacings and quark masses, and with a variety of current operators.
\item{\bf Timeline:} \hfill [last updated May 2024]
\begin{itemize}
    \item{\bf 2021} Lattice calculation of nucleon energy-momentum tensor form factors using overlap fermions, including calculation of momentum fractions and angular momentum fractions for the quarks and glue.
    \item{\bf 2022-2023} Calculations of the glue part of the trace anomaly form factors for pion, $\rho$ meson, and nucleon using domain-wall fermion ensemble and valence overlap fermions at near-physical pion mass.
    \item{\bf 2024-2027} Calculations of the full gravitational form factors of the hadron EMT, i,e, A, B, D, and quark scalar form factors in addition to the trace anomaly form factors. 
\end{itemize}
\end{itemize}


% Keh-Fei Liu
\subsection{Neutrino-nucleon Scattering from the Hadronic Tensor}
\begin{itemize}
    \item{\bf Motivation:} The formalism of hadronic tensor in lattice
      QCD~\cite{Liu:1993cv,Liu:1999ak} offers an opportunity to
      directly evaluate the cross-sections of neutrino-nucleon
      scattering in the elastic and resonance regions and potentially
      in the shallow- and deep-inelastic regions as well. The lattice
      calculation of the neutrino-nucleon scattering cross-sections
      can be an input for the neutrino-nucleus scattering
      calculation. The latter is needed for neutrino experiments at
      DUNE to provide theoretical constraints when analyzing data from
      neutrino oscillation experiments. 
    \item{\bf Long term goal:} Calculate precise neutrino-nucleon
      scattering cross-sections in all the neutrino energy regions
      from the threshold to 7 GeV. This should help improve the
      theoretical understanding of the neutrino-nucleus scattering
      cross-sections and maximize the discovery potential of neutrino
      oscillation experiments. 
    \item{\bf Method:}  Calculation of matrix elements using four-point
      correlation functions with vector and axial two-current
      insertions to the nucleon propagator at different space-time
      positions to obtain the Euclidean hadronic tensor. Bayesian
      reconstruction will be employed to solve the inverse problem and
      obtain the spectral decomposition for the hadronic tensor in the
      Minkowski space. 
\item{\bf Timeline:} \hfill [last updated May 2023]
    \begin{itemize}
    \item{\bf 1993,1999} The hadronic tensor was formulated in the
      Euclidean path-integral formalism~\cite{Liu:1993cv}. A new
      parton degree of freedom -- connected sea partons, was
      discovered and the operator product expansion was
      obtained~\cite{Liu:1999ak}. 
  \item{\bf 2017} The parton evolution equation is generalized to include
    the connected-sea partons~\cite{Liu:2017lpe}. 
    \item{\bf 2019} First exploratory calculation of the hadronic tensor
      was carried out using Backus-Gilbert, Maximum Entropy and
      Bayesian Reconstruction methods are studied in solving the
      inverse problem~\cite{Liang:2019frk}. 
    \item{\bf 2020} Plenary talk at 2019 Lattice Conference which checked
      the higher-twist contribution and verified that the hadronic
      tensor matrix element for the elastic scattering is the product
      of elastic form factors from the three-point
      correlators~\cite{Liang:2020sqi}. 
    \item{\bf 2020} It is shown that the parton degrees of freedom from
      the hadronic tensor are the same as those in the quasi- and
      pseudo-PDFs~\cite{Liu:2020okp}. 
    \item{\bf 2021} The connected-sea partons have been incorporated in
      the global analysis of PDFs for the first
      time~\cite{Hou:2022ajg}
    \item{\bf 2023} Calculation of nucleon elastic and $N\to N^*$
      transition form factors and corresponding longitudinal helicity
      amplitude are completed. A manuscript is being prepared for
      publication. 
    \item{\bf 2024} Calculation of the $\Delta$ contribution in the
      neutrino-nucleon scattering. Explore large momentum transfer to
      study the shallow- and deep-inelastic scattering regions. 
    \item{\bf 2025-2026} High statistics determinations of the resonance
      contributions on multiple ensembles and exploration of 
    the unpolarized, longitudinal and transversely polarized PDFs.
    \item{\bf 2027-} Exploration of generalized parton distribution
      functions through off-forward matrix elements. 
\end{itemize}
\end{itemize}


% William Jay
\subsection{Hadronic Tensor in the Resonance Region}
\begin{itemize}
    \item{\bf Motivation:} The resonant structure of hadrons in kinematic regions with many open channels is a difficult non-perturbative problem in QCD with relevance to accelerator-based neutrino experiments like DUNE as well as the experimental program at the electron-ion collider.
    Recent theoretical and algorithmic advance in lattice QCD make calculations of the hadronic tensor for fully inclusive resonant scattering a realistic possibility for the first time.
    \item{\bf Long term goal:}
    Benchmarking lattice QCD techniques against the well-measured electromagnetic resonant structure functions of the nucleon.
    QCD predictions for the axial resonant structure functions of the nucleon, to ground our understanding of neutrino-nucleus scattering in QCD.
    \item{\bf Method:} Proof-of-concept calculations are underway using the simpler system of the pion instead of the nucleon.
	The calculation involves four-point correlations functions, which are analytically continued from Euclidean space using novel spectral reconstruction techniques.
\item{\bf Timeline:} \hfill [last updated May 2024]
\begin{itemize}
    \item{\bf 2017} Modern theoretical understanding of fully inclusive scattering via smeared spectral functions.~\cite{Hansen:2017mnd}
    \item{\bf 2019} Algorithm for practical spectral reconstruction.~\cite{Hansen:2019idp}
    \item{\bf 2023} Improved algorithm for spectral reconstruction, with rigorously quantified systematic uncertainties.
    \item{\bf 2021-2024} Ongoing proof-of-concept calculations for the electromagnetic hadronic tensor of the pion underway.
    \item{\bf 2027} First QCD calculations of the hadronic tensor of the nucleon.
\end{itemize}
\end{itemize}

\section{Nuclear structure}\label{sec:nucstruct}

% Will Detmold
\subsection{Nuclear and hyper-nuclear interactions}

\begin{itemize}
    \item{\bf Motivation:} Nuclear forces at high density are relevant for understanding  the internal structure  of neutron stars and the dynamics of their mergers. Determinations of these forces are relevant to astrophysical observations from NICER and LIGO/Virgo and in experiments aiming to constrain  these interactions such as CREX/PREX at JLab.
   
    \item{\bf Long term goal:} Fundamental understanding of few-nucleon interactions from QCD to provide a predictive framework for nuclear physics. 
    
    \item{\bf Method:} Scattering
      amplitudes computed from finite-volume energy spectrum. Computation of hadronic two-point functions
      using a large set of multi-hadron operators.
      
\item{\bf Timeline:} \hfill [last updated May 2023]
\begin{itemize}
   \item{\bf 2006} First QCD calculation of $NN$ interactions.
    \item{\bf 2012} Calculation of hyperon-nucleon scattering.
    \item{\bf 2013} First calculation of nuclear and hypernuclear bindings up to $A=4$.
    \item{\bf 2019} Development of automatic code-generation for multi-nucleon correlation functions, enabling much more sophisticated 
    \item{\bf 2021} Large variational study of $NN$ interactions.
    \item{\bf 2025} Three-nucleon spectrum determined using large operator set.
    \item{\bf 2027} Fully-controlled calculations of hyperon-nucleon scattering phase shifts
    
\end{itemize}
\end{itemize}


% Will Detmold
\subsection{Structure of light nuclei}

\begin{itemize}  
    \item{\bf Motivation:} Much is known about the structure of nuclei, but other aspetcs are mysterious when viewed from the point of view of QCD. The recent interest in short-range correlations in nuclei and their relation to the EMC effect has spawned many experiments at JLab and elsewhere; having a QCD understanding of this physics is crucial. 
    
    \item{\bf Long term goal:} Describe nuclear structure from the underlying Standard Model. 
    
    \item{\bf Method:} Matrix elements corresponding to appropriate operators computed using background field methods.
      
\item{\bf Timeline:} \hfill [last updated May 2023]
\begin{itemize}
    \item{\bf 2014} First QCD calculation of magnetic moments and polarizabilities of light nuclei.
    \item{\bf 2015} QCD calculation of the slow-neutron capture cross section, $np\to d\gamma$.
\item{\bf 2018} Calculation of gluon momentum fraction in light nuclei.
    \item{\bf 2020} Calculation of the quark momentum fraction in light nuclei including $^3$He. 
    \item{\bf 2025} Calculation of EM form factors of nuclei.
\end{itemize}
\end{itemize}

\subsection{Nuclear matrix elements for intensity frontier experiments}

\begin{itemize}  
    \item{\bf Motivation:} Many intensity-frontier experiments use nuclei as targets in order to increase cross-sections. These experiments include dark-matter direct-detection experiments such as XENON-100, neutrino-nucleus interaction experiments such as DUNE and current and future neutrinoless double-$\beta$ decay search experiments including nEXO.
    
    \item{\bf Long term goal:}  Provide Standard Model nuclear matrix elements for interpretation of intensity frontier experiments.
    
    \item{\bf Method:} 
    Calculation of nuclear matrix elements using three-point correlation functions.
    
\item{\bf Timeline:} \hfill [last updated May 2023]
\item{\bf 2016} Calculation of neutrinoful double-$\beta$ decay matrix elements revealing importance of isotensor axial polarizability.
\begin{itemize}
    \item{\bf 2018} Calculation of scalar matrix elements in light nuclei relevant for dark matter interactions 
    \item{\bf 2021} Calculation of the axial charge of the triton, providing two-body EFT input for neutrino-nucleus scattering.
    \item{\bf 2024} Calculation of short- and long-distance neutrinoless double-$\beta$ decay matrix elements for $nn\to pp$.
\end{itemize}
\end{itemize}


\section{Extreme matter}\label{sec:extreme}

% Peter Petreczky
\subsection{QCD phase diagram}\label{sec:qcdphase}
\begin{itemize}
  \item{\bf Motivation:} The aim of ultra-relativistic heavy ion
    collisions is to study strongly interacting matter at high
    temperatures. There is a large ongoing experimental program at
    RHIC in BNL and LHC at CERN dedicated to this. Lattice QCD
    calculations provide the necessary input for theoretical
    interpretation of heavy ion experiments. 

 \item{\bf Long term goal:} The goal of lattice QCD calculations is to
      map out the QCD phase diagram as function of the temperature at
      zero and moderately high baryon density, calculate the QCD
      equation of state and estimate spectral functions that are
      needed for heavy flavor and electromagnetic probes of the matter
      produced in ultra-relativistic heavy ion
      collisions. \item{\bf Method:} Calculations are performed using
      highly improved staggered quark (HISQ) action in 2+1 flavor QCD
      with physical strange quark mass, $m_s$ and various light quark
      masses, $m_l$,  including its physical value of $m_l=m_s/27$. In
      some case calculations are performed for larger value of
      $m_l$. For the calculations of the spectral function sometimes
      values $m_l=m_s/5$ are used, while for exploring the chiral
      aspects of the transition smaller than physical values of $m_l$
      are used. 

 \item{\bf Timeline:} \hfill [last updated March 2024]
\begin{itemize}
   \item{\bf 2011-2014} The HotQCD collaboration successfully determined
     the chiral crossover temperature and QCD equation of state in the
     continuum limit. 
   \item{\bf 2014-2024} The HotQCD collaboration determined the chiral
     crossover temperature, the QCD equation of state and fluctuations
     of conserved charges at non-zero density, which was essential
     input for the RHIC BES program. In particular, these results have
     been used by the BEST DOE topical collaboration. The dependence
     of the chiral transition temperature on the chemical potential 
     was determined both for the physical value of the light quark masses
     and for $m_l \rightarrow 0$ limit.
   \item{\bf 2019-2024} Calculations of the bottomonium properties and the
     heavy quark potential have been performed at nearly physical
     light quark masses on lattices with temporal extent
     $N_{\tau}=12$. The heavy quark diffusion coefficient has been
     calculated in 2+1 flavor QCD for the first time at
     $m_l=m_s/5$. All these calculations are important for the
     interpretation of the open and hidden heavy flavor production in
     heavy ion experiments at RHIC and LHC, including STAR, sPHENIX,
     ALICE, CMS and ATLAS.
   \item{\bf 2024-2027} Our goal in the future is to calculate quarkonium
     in-medium properties and the heavy quark diffusion coefficient in
     the continuum limit for physical light quark masses, as well as extending these
     calculations to higher temperatures.
   \item{\bf 2024-2035} Heavy flavor probes will be one of the driving
     themes of heavy ion experiments at RHIC and LHC. The sPHENX and
     STAR experiments will collect new data in 2024-2025. The analysis
     of these data will probably last till 2029. The LHC heavy ion
     experiments are planned to run till at least 2035. 
\end{itemize}
\end{itemize}


%\section{Energy Frontier}\label{sec:energy}
\section{Beyond the Standard Model}\label{sec:BSM}

\subsection{Composite Higgs Models}

% Anna Hasenfratz


\begin{itemize}
\item{ \bf Motivation:} Gauge theories with massless fermions and large
  number of flavors, or in higher fermion representations, become
  near-conformal with the signal of an emergent dilaton. The effective
  field theory description of this dilaton is an important challenge
  to understand as one of the most important clues to the conformal
  window. 
  \item { \bf Long term goal:} Understanding the viability whether the
     observed 125 GeV Higgs boson can be understood as an excitation of
     a new strongly interacting sector
\item {\bf Method:} Large scale numerical simulations of (near)-
     conformal gauge theories or theories close to an IRFP.  
     The effective field theory description of these systems 
     is an important challenge to understand as one of the most
    significant clues to the conformal window. 
\item{\bf Timeline:} \hfill [last updated May 2023]
\begin{itemize}
   \item{\bf 2014-2022} USQCD groups were involved in studies of several possible composite Higgs systems: SU(3) gauge with 8 fundamental flavors, the SU(3) two flavor sextet model, the SU(3)  mass-split system with four light and six heavy fermions, and several SU(4) gauge models with fermions in multiple representations \cite{Fodor:2020niv,LatticeStrongDynamics:2020uwo,Hasenfratz:2023sqa,LSD:2023uzj}. 
   \item{\bf 2023-2024} Each of the above systems are still under active investigation. In many cases the exact nature of the infrared limit is still uncertain. Further investigations with improved actions that allow the reach of stronger couplings where the conformal fixed point (whether it is below or above the system investigated) can be reached.
   \item{\bf 2024-2027} Beyond investigating a potentially light scalar particle, many phenomenologically relevant questions remain. A partial list is given here:
   \begin{itemize}
       \item Large baryon anomalous dimension is essential for partial compositness. So far this has been done  only in a 2-representation model. \cite{Hasenfratz:2023sqa}
       \item Updated $S$ and new $T$ parameter calculations in composite Higgs models.
       \item Compute the scalar form factor of the isosinglet meson in composite Higgs theories. Two methods are available from lattice QCD: direct calculation of the scalar form factor or indirectly using the Feynman-Hellman theorem.
       \item New vector or scalar resonances in diboson production would be very strong evidence of Higgs sector compositeness. Perform calculations of $W^+ Z$ and $W^+ W^-$ scattering in a composite Higgs theory.
   \end{itemize}
\end{itemize}
\end{itemize}



% Anna Hasenfratz
\subsection{Renormalization group properties of gauge-fermion systems }
\begin{itemize}
    \item { \bf Motivation:}
     Understanding the nature of SU(3) gauge theories and how infra-red
     fixed points in many flavor systems arise are important for composite Higgs models. The anomalous dimension of composite fermions is needed to explain fermion mass generation of Standard Model fermions.  
    \item {\bf Long term goal:} Studying SU(N) gauge fermion-systems and explore novel gradient flow methods
    \item {\bf Method:}  Generation dynamical gauge field configurations with
     SU(3) gauge group and $N_f$ fundamental flavors. Calculation of
     the step-scaling function, the discrete analog of the
     renormalization group (RG) $\beta$ function. In addition we
     explore the connection of gradient flow to the RG flow and propose
     a new method to calculate the continuous $\beta$ function,
     extract anomalous dimensions, and obtain renormalization Z factors
\item{\bf Timeline:} \hfill [last updated May 2023]
\begin{itemize}
    \item{\bf 2016-2020} Theoretical development of gradient flow renormalization scheme as described under "QCD $\Lambda$ parameter \dots " section, and  the continuous $\beta$ function (CBF) method \cite{Fodor:2017die,Hasenfratz:2019hpg}.
    \item{\bf 2019-23} Development of a new renormalization scheme for composite fermions \cite{Hasenfratz:2022wll}
    \item{\bf 2023-27} Investigate near conformal systems and establish the critical flavor where the conformal window opens. Calculate anomalous dimensions for mesons and baryons. These investigations tie in closely with the bootstrap study of conformal systems. 
\end{itemize}
\end{itemize}




\subsection{Cosmic Frontier}

\subsubsection{Anomalies in Galaxy Formation}

\begin{itemize}
   \item{ \bf Motivation:} Observed galaxies today do not agree with formation models based on $\Lambda$CDM: cusp-core problem, bugleless disks, \dots Recent JWST observation of massive galaxies in early universe (2208.01611) also a problem.  Modifying $\Lambda$CDM adding dark matter self-interaction may solve these problems.
   \item {\bf Long term goal:} Compute composite dark-matter self-interaction.
   \item{ \bf Method:} Lattice QFT techniques for computing elastic $2 \to 2$ scattering of stable hadrons is well-developed.
\item{\bf Timeline:} \hfill [last updated May 2023]
\begin{itemize}
   \item {\bf 1986} L\"uscher proposed method to extract elastic scattering phase shifts.
    \item {\bf -2023} Quark mass dependence of hadron elastic scattering parameters in SU(3) gauge theories (dark QCD).
   \item{\bf 2021} Launch of JWST. 
   \item {\bf 2023-2025} Baryon-Baryon scattering in SU(4) gauge theories (stealth dark matter).
   \item {\bf 2023-2027} Observation plus astrophysical simulation may provide convincing evidence for modification of $\Lambda$CDM.
\end{itemize}
\end{itemize}

\subsubsection{Space-based Gravitational Wave Detection}

\begin{itemize}
   \item { \bf Motivation:} Stochastic gravitational waves detectible by space-based laser interferometer gravitational wave detectors are an observable consequence of any first-order finite temperature phase transitions that may have occurred in the early universe as it expanded and cooled.  First-order finite temperature phase transitions are a generic feature of many confining gauge theories.  Confining gauge theories are also natural models of composite dark matter as they only need to couple to the Standard Model via gravity.  In certain scenarios, other couplings to the Standard Model may be allowed.
   \item { \bf Long term goal:}  Characterize finite temperature transitions in a wide range of confining gauge theories.
   \item {\bf Method:} Lattice methods of studying QCD thermodynamics can be applied to other theories.
\item{\bf Timeline:} \hfill [last updated May 2023]
\begin{itemize}
   \item { \bf 1990-2023} Phase diagram of SU(3) gauge theories with $N_f=2+1$ ongoing, \textit{i.e.}\ Columbia plot.
   \item{ \bf 2017} LLR Method for computing density of states (1710.06250).
   \item{ \bf 2021-2023} Columbia plot for SU(4) gauge theories with $N_f=2+2$. 
   \item{ \bf 2023-2024} Characterizing SU(4) first-order transitions using LLR method. 
   \item{ \bf 2037} Launch of The Laser Interferometer Space Antenna (LISA)
\end{itemize}
\end{itemize}


\subsection{Theoretical developments}

% Simon Catterall
\subsubsection{Lattice supersymmetry}
\begin{itemize}
\item{\bf Motivation:}
USQCD collaborators are pursuing a program exploring ${\cal N}=4$ super Yang-Mills
theory. A lattice formulation of this theory retaining (some) exact supersymmetry was formulated
more than a decade ago. 
\item {\bf Long term goal:} Numerical simulations of this theory can explore regimes
that are hard or impossible to study using other methods eg. the strong coupling dynamics
with a finite number of colors. In particular can use SYM simulations to prove quantum gravity using holography.
\item{\bf Timeline:} \hfill [last updated March 2024]
   \begin{itemize}
   \item{\bf 2020-21} Optimized parallel code developed based on the 
   MILC codebase. This code is publicly available and has been used to investigate phase structure at strong coupling \cite{Catterall:2020lsi}.
   \item{\bf 2022-23} Computations of correlators of Polyakov lines used to calculate static potential. Good agreement with holography \cite{Catterall:2023tmr}.
\end{itemize}
\end{itemize}

\subsubsection{Quantum field theories on hyperbolic space.}

\begin{itemize}
   \item{\bf Motivation:} USQCD collaborators have also been exploring the phase structure
and dynamics of QFTs on tessellations of hyperbolic space. 

\item {\bf Long term goal:} Study the influence of gravitational fluctuations in
such models by allowing the tessellation to fluctuate. Monte Carlo simulations can be
performed which sample both the space of matter fields and geometries. 
\item{\bf Timeline:} \hfill [last updated March 2024]
\begin{itemize}
\item {\bf 2020-21} One of the generic take home points from such studies is the fact that the conformal
behavior of the boundary theory for a fixed
tessellation can be seen at leading order in a strong coupling or
large mass expansion of the lattice theory. 
Holography on tessellations of hyperbolic space \cite{Asaduzzaman:2020hjl}.
\item {\bf 2022-23} Simulations of Ising model on hyperbolic space \cite{Asaduzzaman:2021bcw}.
\item {\bf 2023-24} Quantum Ising Model on two dimensional AdS space \cite{Asaduzzaman:2023htk}.
\item {\bf 2023-24} Two dimensional quantum gravity, fermions and holography \cite{Asaduzzaman:2024uue}.
\end{itemize}

\end{itemize}

\subsubsection{Exact lattice anomalies and new mirror models}

\begin{itemize}
\item{\bf Motivation:} Symmetric mass generation (SMG), a phenomenon where fermion masses are generated  {\it without} breaking any exact lattice
symmetries is a relatively new phenomenon, studied extensively in condensed matter systems and in 4 dimensions as well \cite{Butt:2021koj,Hasenfratz:2022qan}. It is
now understood that one necessary condition for SMG to occur is that all 
't Hooft anomalies of the theory must cancel. In the case of
staggered fermions, these anomalies are discrete and gravitational in origin and hence
had not been considered previously \cite{Catterall:2022jky}. 
\item {\bf Long term goal:} Embed Weyl fields into reduced K\"{a}hler-Dirac/staggered fields.
Construct mirror models for latter using SMG to gap
out mirror fermions. Map to staggered fermions and determine whether 
continuum limit yields a chiral gauge theory. 
\item {\bf Timeline:}\hfill [last updated March 2024]
\begin{itemize}
    \item {\bf 2023-24} Lattice regularization of reduced K\"{a}hler-Dirac fields and
    connection to chiral fermions \cite{Catterall:2023nww}
    \item {\bf 2024-25} SMG for gauged sp(4) model. Simulations in progress.
    \item {\bf 2024-25} Gauging staggered fermion shift symmetries. In progress.
    
\end{itemize}
\end{itemize}


% Richard Brower
\subsubsection{Lattice Field Theory on Curved Manifolds}
\begin{itemize}
   \item{ \bf Motivation:}
 By placing Lattice Field theory on constant curvature, de Sitter spheres
${\mathbb S}^d$, Cylinders ${ \mathbb R} \times S^{d-1}$ and Euclidian Anti de Sitter
${\mathbb A}d\mathbb{S}^{d+1}$ spaces~\cite{Brower:2019kyh,Brower:2020jqj, Cogburn:2022yyb}, one can
in principle provide a new powerful non-perturbative tool, as
most dramatically exemplified  by lattice QCD, relevant to
both particle and condensed matter physics.  
Recently a fully non-perturbative method has been
found and tested in an map to the strong coupling Ising theory on
${ \mathbb S}^2$~\cite{Brower:2022cwv,Owen:2023}. 
 \item { \bf Long term goal:}
The plan going forward in the next year is to determine the Affine map
for a sequence of 3d and 4d fields theories starting with 3d spin systems and 3d QED before moving to 4d Yang Mills theory. At the same time, we are
introducing these lattice structures in the grid QCD
framework~\cite{Boyle:2016lbp} so that the high performance lattice
algorithm are inherited.  By leveraging state of the art lattice QCD
software, we can follow the lead of Beyond the Standard Model studies
to provide a new approach to near conformal BSM models for composite
Higgs~\cite{LSD:2023uzj} or dark matter~\cite{LatticeStrongDynamics:2020jwi} candidates.  
\end{itemize}

\newpage

% Test 1 2 3

%\appendix

%\bibliographystyle{unsrt}
\bibliographystyle{unsrtnat}
%\bibliographystyle{apsrev}
%\bibliographystyle{plain}
\bibliography{ref.bib}
%\documentclass{article}
\usepackage{graphicx} % Required for inserting images
\usepackage{hyperref}
\title{USQCD theory and experimental time lines}
\author{USQCD Collaboration}
\date{April 2023}

\begin{document}

\maketitle
\tableofcontents
\section{Intensity Frontier}\label{sec:intensity}

\subsection{Muon $g-2$}
\begin{itemize}
    \item[Motivation.] The muon anomalous magnetic moment, or $g-2$, will be measured by the E989 experiment at Fermilab to the one-per=mille precision level and thus offers an exceptional opportunity to test the standard model if theory can match this precision.
    \item[Long term goal.] The Muon $g-2$ Theory Initiative aims to provide a consensus theory value at the same level of precision as the E989 experiment. Only QCD, or hadronic, contributions need improvement to reach this goal.
\item[Method.] Lattice  calculations of correlation functions in QCD for the hadronic vacuum polarization (HVP) (two-point) and hadronic light-by-light scattering (HLbL) (four-point) are combined with perturbative QED at $O(\alpha^2)$ and $O(\alpha^3)$, respectively, to obtain the leading hadronic contributions.
\item[Timeline:]
%\begin{itemize}
    \item[2021] Muon $g-2$ Theory Initiative whitepaper~\cite{Aoyama:2020ynm} released with consensus Standard Model Value based on data-driven HVP and data-driven+lattice HLbL contributions.
    \item[2021] E989 announces first results with 0.48 ppm precision. Combined with BNL 821 (0.54 ppm), yields 4.2 standard deviation discrepancy with theory.
    \item[2021] BMW collaboration computes HVP contribution to 0.75\%, consistent with experiment. 
    \item[2022] USQCD groups Fermilab Lattice/HPQCD/MILC and RBC/UKQCD, $\chi$QCD, Aubin {\it et al.}, and several groups from Europe obtain agreement on the HVP Window which is four standard deviations discrepant with the data-driven value.
    \item[2023] Several groups to update total HVP contribution, including isospin corrections, with sub-percent errors.
    \item[2023] RBC updated HLbL contribution, reaching 12\% error.
    \item[2023] E989 to announce run 2 and 3 combined result, improving precision by a factor of two.
    \item[2024] RBC to reduce error on HLbL contribution by a factor of two.
    \item[2024-25] RBC/UKQCD and FHM reach 1-2 per-mille goal on HVP contribution. 
    \item[2025-26] E989 releases final result with expected precision of $\sim 1.4$ ppm
%\end{itemize}
\end{itemize}


\subsection{$B$-meson anomalies and CKM unitarity tests}
\begin{itemize}
    \item[Motivation.] $B$-flavor physics offers excellent opportunities for stringent tests of the Standard Model, tests that may lead to the discovery of new physics. In conjunction with refined experimental results, precision lattice-QCD calculations yield precise values of the CKM matrix elements, allowing checks of CKM unitarity and tests of lepton universality. Measurements of rare semileptonic $B$-meson decays (such as $B\to\pi\ell^+\ell^-$, $B\to K\ell^+\ell^-$) are promising channels for new-physics searches because their rates are suppressed in the Standard Model.
    \item[Long term goal.] Lattice-QCD calculations go hand-in-hand with experiment.  Thus the long-term goal is to keep pace with the improving experimental precision of each quantity under study.
    \item[Method.] Most calculations work with a combination of hadronic two- and three-point functions, measured at several lattice spacings and quark masses, and with a variety of current operators. 
\item[Timeline:]
%\begin{itemize}
    \item[2014] Calculation of leptonic $B$, $B_s$, $D$, and $D_s$ decay constants to sub-percent accuracy.
    \item[2015] First measurement by CMS/LHCb of the decay $B_s \to \mu^+\mu^-$.
    \item[2021] Calculation of $R(D*)$ to 4\%.
    \item[2022] Precise calculation of $|V_{cs}|$ with a QCD error comparable to the current experimental precision and of $|V_{cd}|$ with a QCD error twice that of experiment.
    \item[2024] Calculation of $|V_{cb}|$ to 1\% via $B\to D\ell\nu$ and $|V_{ub}$ to 2\% via $B\to \pi\ell\nu$.
    \item[2024] Calculation of $B\to\pi \ell^+\ell^-$ and $B\to K \ell^+\ell^-$ to 2\%.
    \item[2027] Calculation of $|V_{cb}|$ to the sub-percent level via $B\to D^*\ell\nu$.
    \item[2027] Calculation of $R(D*)$ to 1\%.
    \item[??] Belle II
    \item[??] LHCb
    \item[??] BES III
%\end{itemize}
\end{itemize}


\subsection{$K_L - K_S$ mass difference}
\begin{itemize}
    \item[Motivation.] Precisely measured quantity that has sensitivity to new physics at an energy scale of $10^4$ TeV.
    \item[Long term goal.] Calculate $K_L - K_S$ mass difference with
      a precision exceeding the current experimental value of $3.484
      \pm 0.007 \times  10^{-12}$ MeV.
    \item[Method.] Well understood lattice QCD technique with no
      recognized limitation to the control of all systematic
      errors. GIM cancellation is essential and treatment of the
      charmed quark using QCD perturbation theory introduces 36\%
      errors. The lattice calculation is made difficult by the
      requirement of a lattice spacing smaller than the charm quark
      Compton wave length and a volume sufficiently large to accurate
      treat physical mass pions. 
\item[Timeline:]
%\begin{itemize}
    \item[2011] Most accurate KTeV measurement.
    \item[2010-2014] Lattice QCD method devised and first results obtained.
    \item[2014-2021] First results with physical masses and $1/a = 2.38$ MeV obtained, presence of large discretization errors recognized.
    \item[2021-2023] Discretization errors under study and evidence of $a^2$ scaling expected.
    \item[2023-2024] Calculation of the long-distance contribution to
      $\varepsilon_K$ with $1/a = 2.38$ MeV will include extension of
      earlier $\Delta M_K$, calculation increasing statistics.
    \item[2024-2026] Calculation of $\Delta M_K$ with $1/a = 2.8$ GeV
      giving continuum limit result with $< 20$\% errors. 
    \item[2027-2030] Move to $2+1+1$ flavors and $1/a = 3.0$, $4.0$
      GeV and possibly 5 GeV with $<10$\% errors. 
%\end{itemize}
\end{itemize}


\section{Hadron spectroscopy}\label{sec:hadspec}

\subsection{Exotic light quark mesons}
\begin{itemize}
    \item[Motivation.] Observation of putative exotic $J^{PC}$ bosonic
      states. Focus of GlueX@JLab, CLAS12@JLab, COMPASS@LHC. GlueX Run
      II underway through 2023, then Run III starting in 2025 after
      kaon 
PID upgrade.
    \item[Long term goal.] Determine the mass and decay modes of
      putative light quark hybrid and exotic mesons. Extract resonance
      parameters of meson and baryon spectrum. Guide experimental
      searches with predictions of decay couplings.
    \item[Method.] Resonance spectrum extracted from scattering
      amplitudes computed from finite-volume energy spectrum. Cost
      driven by annihilation annihilation quark lines computed on many
      time-slices and computation of hadronic two-point functions
      featuring a large multi-hadron operator basis utilizing the varational method. 
\item[Timeline:]
%\begin{itemize}
    \item[2013] Isovector and isoscalar light quark meson and baryon
      spectrum computed with restriction to single particle basis~\cite{Dudek:2013yja}. Results featured in PDG.
    \item[2015] First computation of resonance parameters from coupled-channel scattering
      amplitudes of $\pi\pi/\bar{K}K$ \cite{Wilson:2015dqa}.
    \item[2019] Phenomenological extraction of $\pi_1$ resonance from 
      parameters from partial waves of $\eta(')\pi$ measured by COMPASS~\cite{JPAC:2018zyd}.
    \item[2022] First determination of full three-body relativistic scattering amplitude~\cite{Hansen:2020otl}
    \item[2022] Prediction for light-quark isovector $J^{PC}=1^{-+}$ published~\cite{Woss:2020ayi}.
    \item[2024] Putatitive light-quark isoscalar exotic meson ($\eta_1$)  resonance parameters.
    \item[2025] GlueX results based on runs I and II. 
    \item[2027] GlueX results based on combined data through run III.
%\end{itemize}
\end{itemize}


\section{Hadron structure}\label{sec:hadstruct}


\section{Nuclear structure}\label{sec:nucstruct}



\section{Energy Frontier}\label{sec:energy}




\bibliographystyle{unsrt}
\bibliography{ref.bib}

\end{document}


\end{document}
