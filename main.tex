\documentclass{article}
\usepackage{graphicx} % Required for inserting images
\usepackage{hyperref}
\title{USQCD theory and experimental time lines}
\author{USQCD Collaboration}
\date{April 2023}

\begin{document}

\maketitle
\tableofcontents
\section{Intensity Frontier}\label{sec:intensity}

% test

\subsection{Muon $g-2$}
\begin{itemize}
    \item[Motivation.] The muon anomalous magnetic moment, or $g-2$, will be measured by the E989 experiment at Fermilab to the one-per=mille precision level and thus offers an exceptional opportunity to test the standard model if theory can match this precision.
    \item[Long term goal.] The Muon $g-2$ Theory Initiative aims to provide a consensus theory value at the same level of precision as the E989 experiment. Only QCD, or hadronic, contributions need improvement to reach this goal.
\item[Method.] Lattice  calculations of correlation functions in QCD for the hadronic vacuum polarization (HVP) (two-point) and hadronic light-by-light scattering (HLbL) (four-point) are combined with perturbative QED at $O(\alpha^2)$ and $O(\alpha^3)$, respectively, to obtain the leading hadronic contributions.
\item[Timeline:]
%\begin{itemize}
    \item[2021] Muon $g-2$ Theory Initiative whitepaper~\cite{Aoyama:2020ynm} released with consensus Standard Model Value based on data-driven HVP and data-driven+lattice HLbL contributions.
    \item[2021] E989 announces first results with 0.48 ppm precision. Combined with BNL 821 (0.54 ppm), yields 4.2 standard deviation discrepancy with theory.
    \item[2021] BMW collaboration computes HVP contribution to 0.75\%, consistent with experiment. 
    \item[2022] USQCD groups Fermilab Lattice/HPQCD/MILC and RBC/UKQCD, $\chi$QCD, Aubin {\it et al.}, and several groups from Europe obtain agreement on the HVP intermediate window which is four standard deviations discrepant with the data-driven value.
    \item[2023] Several groups to update total HVP contribution, including isospin corrections, with sub-percent errors.
    \item[2023] RBC updated HLbL contribution, reaching 12\% error.
    \item[2023] E989 to announce run 2 and 3 combined result, improving precision by a factor of two.
    \item[2024] RBC to reduce error on HLbL contribution by a factor of two.
    \item[2024-25] RBC/UKQCD and FHM reach 1-2 per-mille goal on HVP contribution. 
    \item[2025-26] E989 releases final result with expected precision of $\sim 1.4$ ppm
%\end{itemize}
\end{itemize}


\subsection{$B$-meson anomalies and CKM unitarity tests}
\begin{itemize}
    \item[Motivation.] $B$-flavor physics offers excellent opportunities for stringent tests of the Standard Model, tests that may lead to the discovery of new physics. In conjunction with refined experimental results, precision lattice-QCD calculations yield precise values of the CKM matrix elements, allowing checks of CKM unitarity and tests of lepton universality. Measurements of rare semileptonic $B$-meson decays (such as $B\to\pi\ell^+\ell^-$, $B\to K\ell^+\ell^-$) are promising channels for new-physics searches because their rates are suppressed in the Standard Model.
    \item[Long term goal.] Lattice-QCD calculations go hand-in-hand with experiment.  Thus the long-term goal is to keep pace with the improving experimental precision of each quantity under study.
    \item[Method.] Most calculations work with a combination of hadronic two- and three-point functions, measured at several lattice spacings and quark masses, and with a variety of current operators. 
\item[Timeline:]
%\begin{itemize}
    \item[2014] Calculation of leptonic $B$, $B_s$, $D$, and $D_s$ decay constants to sub-percent accuracy.
    \item[2015] First measurement by CMS/LHCb of the decay $B_s \to \mu^+\mu^-$.
    \item[2021] Calculation of $R(D*)$ to 4\%.
    \item[2022] Precise calculation of $|V_{cs}|$ with a QCD error comparable to the current experimental precision and of $|V_{cd}|$ with a QCD error twice that of experiment.
    \item[2024] Calculation of $|V_{cb}|$ to 1\% via $B\to D\ell\nu$ and $|V_{ub}$ to 2\% via $B\to \pi\ell\nu$.
    \item[2024] Calculation of $B\to\pi \ell^+\ell^-$ and $B\to K \ell^+\ell^-$ to 2\%.
    \item[2027] Calculation of $|V_{cb}|$ to the sub-percent level via $B\to D^*\ell\nu$.
    \item[2027] Calculation of $R(D*)$ to 1\%.
    \item[??] Belle II
    \item[??] LHCb
    \item[??] BES III
%\end{itemize}
\end{itemize}


\subsection{$K_L - K_S$ mass difference}
\begin{itemize}
    \item[Motivation.] Precisely measured quantity that has sensitivity to new physics at an energy scale of $10^4$ TeV.
    \item[Long term goal.] Calculate $K_L - K_S$ mass difference with
      a precision exceeding the current experimental value of $3.484
      \pm 0.007 \times  10^{-12}$ MeV.
    \item[Method.] Well understood lattice QCD technique with no
      recognized limitation to the control of all systematic
      errors. GIM cancellation is essential and treatment of the
      charmed quark using QCD perturbation theory introduces 36\%
      errors. The lattice calculation is made difficult by the
      requirement of a lattice spacing smaller than the charm quark
      Compton wave length and a volume sufficiently large to accurate
      treat physical mass pions. 
\item[Timeline:]
%\begin{itemize}
    \item[2011] Most accurate KTeV measurement.
    \item[2010-2014] Lattice QCD method devised and first results obtained.
    \item[2014-2021] First results with physical masses and $1/a = 2.38$ MeV obtained, presence of large discretization errors recognized.
    \item[2021-2023] Discretization errors under study and evidence of $a^2$ scaling expected.
    \item[2023-2024] Calculation of the long-distance contribution to
      $\varepsilon_K$ with $1/a = 2.38$ MeV will include extension of
      earlier $\Delta M_K$, calculation increasing statistics.
    \item[2024-2026] Calculation of $\Delta M_K$ with $1/a = 2.8$ GeV
      giving continuum limit result with $< 20$\% errors. 
    \item[2027-2030] Move to $2+1+1$ flavors and $1/a = 3.0$, $4.0$
      GeV and possibly 5 GeV with $<10$\% errors. 
%\end{itemize}
\end{itemize}


\section{Hadron spectroscopy}\label{sec:hadspec}

\subsection{Exotic light quark mesons}
\begin{itemize}
    \item[Motivation.] Observation of putative exotic $J^{PC}$ bosonic
      states. Focus of GlueX@JLab, CLAS12@JLab, COMPASS@LHC. Kaon PID upgrade of GlueX and recommissioning starting 2020. Phase II run underway through 2025. Transition to JLab Eta Factory.
      %Phase III possibly starting in 2027.
    \item[Long term goal.] Determine the mass and decay modes of
      putative light quark hybrid and exotic mesons. Extract resonance
      parameters of meson and baryon spectrum. Guide experimental
      searches with predictions of decay couplings.
    \item[Method.] Resonance spectrum extracted from scattering
      amplitudes computed from finite-volume energy spectrum. Cost
      driven by annihilation quark lines computed on many
      time-slices and computation of hadronic two-point functions
      featuring a large multi-hadron operator basis utilizing the variational method. 
\item[Timeline:]
%\begin{itemize}
    \item[2013] Isovector and isoscalar light quark meson and baryon
      spectrum computed with restriction to single particle basis~\cite{Dudek:2013yja}. Results featured in PDG.
    \item[2015] First computation of resonance parameters from coupled-channel scattering
      amplitudes of $\pi\pi/\bar{K}K$ \cite{Wilson:2015dqa}.
    \item[2019] Phenomenological extraction of $\pi_1$ resonance from 
      parameters from partial waves of $\eta(')\pi$ measured by COMPASS~\cite{JPAC:2018zyd}.
    \item[2022] First determination of full three-body relativistic scattering amplitude~\cite{Hansen:2020otl}
    \item[2022] Prediction for light-quark isovector $J^{PC}=1^{-+}$ published~\cite{Woss:2020ayi}.
    \item[2024] Putative light-quark hybrid meson multiplet resonance parameters.
    \item[2025] GlueX Phase II completed. Begin analysis of runs I and II. 
    \item[2027] GlueX results based on combined data through run II.
%\end{itemize}
\end{itemize}


\subsection{Nuclear and hyper-nuclear interactions}

\begin{itemize}
    \item[Motivation.] Nuclear forces at high density are relevant for understanding  the internal structure  of neutron stars and the dynamics of their mergers. Determinations of these forces are relevant to astrophysical observations from NICER and LIGO/Virgo and in experiments aiming to constrain  these interactions such as CREX/PREX at JLab.
    
    \item[Long term goal.] Fundamental understanding of few-nucleon interactions from QCD to provide a predictive framework for nuclear physics. 
    
    \item[Method.] Scattering
      amplitudes computed from finite-volume energy spectrum. Computation of hadronic two-point functions
      using a large set of multi-hadron operators.
      
\item[Timeline:]
    \item[2006] First QCD calculation of $NN$ interactions.
    \item[2012] Calculation of hyperon-nucleon scattering.
    \item[2013] First calculation of nuclear and hypernuclear bindings up to $A=4$.
    \item[2019] Development of automatic code-generation for multi-nucleon correlation functions, enabling much more sophisticated 
    \item[2021] Large variational study of $NN$ interactions.
    \item[2025] Three-nucleon spectrum determined using large operator set.
    \item[2027] Fully-controlled calculations of hyperon-nucleon scattering phase shifts
    
\end{itemize}


\section{Hadron structure}\label{sec:hadstruct}


\section{Nuclear structure}\label{sec:nucstruct}

\subsection{Structure of light nuclei}

\begin{itemize}  
    \item[Motivation.] Much is known about the structure of nuclei, but other aspetcs are mysterious when viewed from the point of view of QCD. The recent interest in short-range correlations in nuclei and their relation to the EMC effect has spawned many experiments at JLab and elsewhere; having a QCD understanding of this physics is crucial. 
    
    \item[Long term goal.] Describe nuclear structure from the underlying Standard Model. 
    
    \item[Method.] Matrix elements corresponding to appropriate operators computed using background field methods.
      
\item[Timeline:]
    \item[2014] First QCD calculation of magnetic moments and polarizabilities of light nuclei.
    \item[2015] QCD calculation of the slow-neutron capture cross section, $np\to d\gamma$.
\item[2018] Calculation of gluon momentum fraction in light nuclei.
    \item[2020] Calculation of the quark momentum fraction in light nuclei including $^3$He. 
    \item[2025] Calculation of EM form factors of nuclei.
\end{itemize}

\subsection{Nuclear matrix elements for intensity frontier experiments}

\begin{itemize}  
    \item[Motivation.] Many intensity-frontier experiments use nuclei as targets in order to increase cross-sections. These experiments include dark-matter direct-detection experiments such as XENON-100, neutrino-nucleus interaction experiments such as DUNE and current and future neutrinoless double-$\beta$ decay search experiments including nEXO.
    
    \item[Long term goal.]  Provide Standard Model nuclear matrix elements for interpretation of intensity frontier experiments.
    
    \item[Method.] 
    Calculation of nuclear matrix elements using three-point correlation functions.
    
\item[Timeline:]
\item[2016] Calculation of neutrinoful double-$\beta$ decay matrix elements revealing importance of isotensor axial polarizability.
    \item[2018] Calculation of scalar matrix elements in light nuclei relevant for dark matter interactions 
    \item[2021] Calculation of the axial charge of the triton, providing two-body EFT input for neutrino-nucleus scattering.
    \item[2024] Calculation of short- and long-distance neutrinoless double-$\beta$ decay matrix elements for $nn\to pp$.
\end{itemize}




\section{Extreme matter}\label{sec:extreme}


\section{Energy Frontier}\label{sec:energy}




\bibliographystyle{unsrt}
\bibliography{ref.bib}

\end{document}
